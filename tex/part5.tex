\documentclass{article}
\usepackage{CJK}
\usepackage{url}

\begin{document}
\begin{CJK*}{Bg5}{ming} 
這七篇文章包含一些在~comp.unix.questions 和~comp.unix.shell 常見到的問
題。請不要再問這些問題,因為這些問題已經回答過太多次了。但也請不要因為
有人問這些問題而發火,因為他們可能尚未讀過這些文章。

This collection of documents is Copyright (c) 1994, Ted Timar, except 
Part 6, which is Copyright (c) 1994, Pierre Lewis and Ted Timar.

All rights reserved. Permission to distribute the collection is 
hereby granted providing that distribution is electronic, no money is 
involved, reasonable attempts are made to use the latest version and 
all credits and this copyright notice are maintained.

Other requests for distribution will be considered.

All reasonable requests will be granted.

中文翻譯~by \url{{chenjl,freedom,jjyang}@csie.nctu.edu.tw}
若您對中文翻譯有任何意見請發~e-mail 給~\url{cfaq@csie.nctu.edu.tw}

我們希望這些文件中的資訊能對你有所幫助﹐但是並不保証是正確的。若發生損
害請自行負責。

您可以在~rtfm.mit.edu 的~pub/usenet/news.answers 找到包括此文件在內的
許多~FAQ。 在此目錄下的~FAQ 的名字可在文章的頂端的~"Archive-Name:" 
那一行找到。\footnote{ 在台灣請用~\url{ftp://NCTUCCA.edu.tw/USENET/FAQ},
若您人在交大的話 ~\url{ftp://ftp.csie.nctu.edu.tw/pub/Documents/FAQ} 是從
~CCCA mirror  來的}

此一~FAQ 是以~"unix-faq/faq/part[1-7]" 為名。

這些文章大約分成:
\begin{quote} 
      1.*)一般性的問題 \newline
      2.*)初學者可能會問的基本問題 \newline
      3.*) 中級的問題 \newline
      4.*) 自以為已經知道所有答案的人可能會問的高級問題 \newline
      5.*) 關於各種~shell 的問題 \newline
      6.*) 各式各樣的~Unix \newline
      7.*) An comparison of configuration management systems (RCS, SCCS).
\end{quote}
這篇文章回答了以下問題:
\tableofcontents

因為這些都是正當合理的問題, 所以在 \url{comp.unix.questions} 或是 
\url{comp.unix.shell} 中。每隔一陣子, 就會有這些問題與答案出現, 緊接著就會
有人對同樣問題一再出現發牢騷。關於~UNIX 代表啥呢? 請參考每月 post 在 
\url{news.announce.newusers} 中名為~"Answers to Frequently Asked Questions" 
的文章。

因為~Unix 有太多不同的種類了, 所以很難保證此文件所提供的答案必然會有
用。在嘗試本文件提供的作法前, 請先讀讀你所使用系統的手冊。若你對答案
有任何建議或更正, 請送~email 給 tmtaimar@isgtec.com.

\setcounter{section}{4}
\section{關於各種~shell 的問題}
\subsection{shell 要如何分類呢?}

	一般來說,shell 可以分成兩類。第一類是由~Bourne shell 衍生出來的包括 
	~sh,ksh,bash,與~zsh。第二類是由~C shell 衍生出來的,包括~csh 與 
	~tcsh。除此之外還有一個~rc,有人認為該自成一類,有人認為該歸類在 
	~Bourne shell。

	把上面的分類法記住,就可以寫出所有~Bourne shell類的~shell 或是所有~C 
	shell 類的~shell 都可用的~script。

\subsection{在一個~shell script 裡面要如何~"include" 另一個~shell script 呢?}

	由~Bourne shell 衍生者(包括~rc)用~"." 這個指令。由~C shell 衍生者則使
        用~"source"。

\subsection{是不是所有~shell 都有~alias,有沒別的替代方案呢? }

	除了~sh 外,主要的~shell 都有~alias,不過用法不大一樣。例如有些~alias 
        不接受參數。

	Bourne shell 類中~shell function 雖然嚴格說來不是~alias,不過也差不多
	了。Shell function 還可以做到一些~alias 所做不到的事情。從~Version 7 
	Unix 的衍生出來的系統(包括 System III 與 4.2BSD) 裡面的~Bourne shell 
	沒有~shell function。System V 與~4.3BSD 的~sh 都有~shell function。

	用~unalias 來去除~alias, 用~unset 來去除~shell function。

\subsection{要如何設定~shell 的變數呢?}

	在~C shell 類的~shell 中用~"set variable=value" 來設定只給這個~shell 
        看的變數,用~"setenv variable value" 來設定環境變數。要去除變數則以 
       ~unset 與~unsetenv 為之。在~Bourne shell 類的~shell 中則用 
       ~"variable=value" 來設定變數,若要將一變數並成環境變數則用~"export 
       VARIABLE\_NAME"。要去除一個變數則用~unset。 

\subsection{要如何得知是否是在一~interactive shell 中?}

	在~C shell 類中,看看~\$prompt 這個變數便知分曉。

	在~Bourne shell 類中,可以看~\$PS1 這個變數,不過最好是檢查~\$- 這個變
	數。假如\$- 裡有~i,那就表示是在~interactive shell 中。作法如下:
\begin{verbatim}
          case $- in
          *i*)    # do things for interactive shell
                  ;;
          *)      # do things for non-interactive shell
                  ;;
          esac
\end{verbatim}
\subsection{那些以~"." 開始其檔名的檔案在各種~shell 中的用途為何?}
     下面的說明或許不盡完整,但應該講的都講了。
\begin{description}
      \item[csh]
\begin{quote}
	有一些版本在系統中所有使用者都必須讀取的~.cshrc 與~.login。不
	過,每一個版本放置這些檔案的地方不大相同。
	\begin{itemize}
        \item  啟動 (依照所列順序):
	  \begin{itemize}
          \item   .cshrc   - 每次都會讀取
          \item   .login   - login shell 才會讀取
	  \end{itemize}

        \item  結束之前:
	  \begin{itemize}
            \item   .logout  - login shells.
	  \end{itemize}

        \item  其他:
	  \begin{itemize}
             \item  .history - 儲存history (根據 \$savehist 的值決定大小).
	  \end{itemize}
	\end{itemize}
\end{quote}

      \item[tcsh]
\begin{quote}
	\begin{itemize}
	  \item 啟動 (依照所列順序):
	  \begin{itemize}
             \item /etc/csh.cshrc - 每次都會讀取
             \item /etc/csh.login - login shells 才會讀取
             \item .tcshrc        - 每次都會讀取
             \item .cshrc         - 假如沒有 .tcshrc 在的話
             \item .login         - login shells 才會讀取
	  \end{itemize}

	  \item 結束之前:
	  \begin{itemize}
             \item .logout        - login shells 才會讀取.
	  \end{itemize}

	  \item 其他:
	  \begin{itemize}
             \item .history       - saves the history (based on \$savehist).
             \item .cshdirs       - saves the directory stack.
	  \end{itemize}
	\end{itemize}
\end{quote}

      \item[sh]
\begin{quote}
	\begin{itemize}
          \item 啟動 (依照所列順序):
	  \begin{itemize}
             \item /etc/profile - login shell 才會讀取
             \item .profile     - login shell 才會讀取
	  \end{itemize}

          \item 結束之前:
	  \begin{itemize}
             \item any command (or script) specified using the command:
                 trap "command" 0
	  \end{itemize}
	\end{itemize}
\end{quote}

      \item[ksh]
\begin{quote}
	\begin{itemize}
	\item 啟動 (依照所列順序):
	  \begin{itemize}
              \item /etc/profile - login shells 才會讀取
              \item .profile     - login shells 才會讀取
              \item \$ENV        - always, if it is set.
	  \end{itemize}

          \item 結束之前:
	  \begin{itemize}
            \item 如果你用了~trap "command" 0,則結束之際會去執行~"command"。
	  \end{itemize}
	\end{itemize}
\end{quote}

      \item[bash]
\begin{quote}
	\begin{itemize}
         \item 啟動 (依照所列順序):
	 \begin{itemize}
           \item   /etc/profile  - login shell 才會讀取
           \item   .bash\_profile - login shell 才會讀取
           \item   .profile      - 如果沒有 .bash\_profile,則會在 login 時讀取
           \item   .bashrc       - interactive non-login shell 才會讀取
           \item   \$ENV          - 若有設定則讀取之
	 \end{itemize}

         \item 結束之前:
	 \begin{itemize}
            \item  .bash\_logout  - login shell 讀取之
	 \end{itemize}
         \item 其他:
	 \begin{itemize}
            \item  .inputrc      - Readline 初始化之時
	 \end{itemize}
	\end{itemize}
\end{quote}
      \item[zsh]
\begin{quote}
	\begin{itemize}
          \item 啟動 (依照所列順序):
	\begin{itemize}
             \item  .zshenv - 若無 -f  就會讀取
             \item .zprofile - login shell 才會讀取
             \item  .zshrc    - interactive shell 若無 -f  就會讀取
             \item .zlogin   - login shell 結束之前
	\end{itemize}

          \item 結束之前:
	\begin{itemize}
             \item .zlogout  - login shell 才會讀取
	\end{itemize}
	\end{itemize}
\end{quote}
      \item[rc]
\begin{quote}
	\begin{itemize}
          \item 啟動 (依照所列順序):
	   \begin{itemize}
             \item .rcrc - login shell 才會讀取
	   \end{itemize}
	\end{itemize}
\end{quote}
\end{description}

\subsection{我想要知道更多各種~shell 的相異之處。請指點明路。}

	你可以用~anonymous ftp 在下列地點取得一份有關~sh、tcsh、ksh、
	bash、zsh 與~rc 的詳細比較。

      \url{ftp://cs.uwp.edu/pub/vi/docs/shell-101.BetaA.Z} 和~
      \url{ftp://utsun.s.u-tokyo.ac.jp/misc/vi-archive/docs/shell-101.BetaA.Z}

	這個檔案中比較了呼叫~shell 時的旗號、程式的文法、輸出/入之轉向、與環
	境變數。不過這個檔案並不討論以 "." 為檔名開頭的檔案以及環境變數與
	函數的繼承。
\end{CJK*}
\end{document}
