\documentclass{letter}
\usepackage{CJK}
\usepackage{url}

\signature{T\^an Koan-S\^{\i}n}

\begin{document}
\begin{CJK*}{Bg5}{ming}

關於 FAQ 中文版:
鑑於  tw.bbs.comp.unix 中一再出現重覆的問題, 我們
(\url{{chenjl,freedom,jjyang}@csie.nctu.edu.tw})
將 comp.unix.questions 的 FAQs 翻譯為中文, 希望能
夠稍稍改善這種情況, 此 faq 中文版置於 
\url{ftp://ftp.csie.nctu.edu.tw/pub/CSIE/contrib/cfaq/unix} 下
亦可由 
\url{gopher://gopher.csie.nctu.edu.tw/11/NCTU/CSIE/center/
ccfaq/unix-faq/unix-faq}
觀之

此 FAQ 原有 7 個 part, 我們只譯了前 6 個, 因為 part 7 的內容
是 RCS 與 SCCS 的比較這兩套 source code version control 的軟
體, 台灣用的人本就不多, 而 FAQ 中的內容又都是一面之詞 FAQ 的
維護人 Ted Timar 也有意要把它拿掉, Timar 先生說:
\begin{quote}
   I will probably take out part7 soon, since it is very biased, and nobody
   has bothered to give the other side of the picture.
\end{quote}
故實無翻譯之必要

其它相關的中文線上資訊:
\begin{enumerate}
\item \url{gopher://gopher.csie.nctu.edu.tw:70/
	00/NCTU/CSIE/center/ccfaq/csiefaq}

        主要是從 csie.help 此 newsgroup 中, 找出常見的問題及解答, 
        再加上中心助教的補充而成. 所以在此文件中, 所有的設定都是
        以交大資工系計中為主. 

\item \url{gopher://gopher.csie.nctu.edu.tw:70/11/NCTU/CSIE/
	center/user_guide}

        交大資工系計中給新生看的電腦工作站基本操作手冊

\item \url{ftp://phi.sinica.edu.tw/pub/aspac/doc/}

      有不少中央研究院計算中心ASPAC計劃所生產出來與 UNIX 相關的中文文件

\item \url{ftp://ftp.csie.nctu.edu.tw/pub/Documents/UNIX-Guide} 與
      \url{ftp://moers2.edu.tw/chinese-pub/documents/UNIX-Guide/V2.4E}

        工研院電通所楊景翔先生<js@v0sun11.ccl.itri.org.tw>所寫的
        ``UNIX初學者使用手冊'',已經由第三波出版社出版,
        無法自 Internet 取得或者雖然拿得到卻無法列印 
        Postscript 檔案的人可以自行在各大書局買到。

\item \url{ftp://ftp.edu.tw//Chinese/YuanInfo/}

	有不少曾瑞源先生關於 Internet, UNIX, Linux 的書, 文章

\item TANet 各大 BBS 的精華區應該都有整理一些東西

\item \url{ftp://ftp.csie.nctu.edu.tw/pub/Chinese/chinese-text/big5-faq} 與
   \url{ftp://ftp.csie.nctu.edu.tw/pub/Chinese/chinese-text/big5-faq.chinese}

   英文版和中文版的 BIG5 環境問答

\end{enumerate}

中文 Unix 書籍:
(歡迎大家提供書目)
\begin{enumerate}
\item UNIX與Internet實用手冊, 白晉源﹑陳曉陽, 倚天資訊股份有限公司出版

	對 TANet 的始用者而言這是本絕佳的入門書
    
\item 和碩有一系列的 Unix 中文書, 由淺到深都有, 不過品質參差不齊, 有幾本
   的內容與台灣常用的 Unix 環境差距太大, 要買得找個懂 Unix 的人陪你挑
   選

\item UNIX 之 Q\&A 入門篇, 許志明編著, 儒林圖書有限公司出版 
        
       是不錯的 UNIX 入門書, 如果校對時用心一點就更好了

\end{enumerate}
\end{CJK*}
\closing{Best regards,}
\end{document}
