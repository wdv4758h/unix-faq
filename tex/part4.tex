\documentclass{article}
\usepackage{CJK}
\usepackage{url}

\begin{document}
\begin{CJK*}{Bg5}{ming} 
這七篇文章包含一些在~comp.unix.questions 和~comp.unix.shell 常見到的問
題。請不要再問這些問題,因為這些問題已經回答過太多次了。但也請不要因為
有人問這些問題而發火,因為他們可能尚未讀過這些文章。

This collection of documents is Copyright (c) 1994, Ted Timar, except 
Part 6, which is Copyright (c) 1994, Pierre Lewis and Ted Timar.

All rights reserved. Permission to distribute the collection is 
hereby granted providing that distribution is electronic, no money is 
involved, reasonable attempts are made to use the latest version and 
all credits and this copyright notice are maintained.

Other requests for distribution will be considered.

All reasonable requests will be granted.

中文翻譯~by \url{{chenjl,freedom,jjyang}@csie.nctu.edu.tw}
若您對中文翻譯有任何意見請發~e-mail 給~\url{cfaq@csie.nctu.edu.tw}

我們希望這些文件中的資訊能對你有所幫助﹐但是並不保証是正確的。若發生損
害請自行負責。

您可以在~rtfm.mit.edu 的~pub/usenet/news.answers 找到包括此文件在內的
許多~FAQ。 在此目錄下的~FAQ 的名字可在文章的頂端的~"Archive-Name:" 
那一行找到。\footnote{ 在台灣請用~\url{ftp://NCTUCCA.edu.tw/USENET/FAQ},
若您人在交大的話 ~\url{ftp://ftp.csie.nctu.edu.tw/pub/Documents/FAQ} 是從
~CCCA mirror  來的}

此一~FAQ 是以~"unix-faq/faq/part[1-7]" 為名。

這些文章大約分成:
\begin{quote} 
      1.*)一般性的問題 \newline
      2.*)初學者可能會問的基本問題 \newline
      3.*) 中級的問題 \newline
      4.*) 自以為已經知道所有答案的人可能會問的高級問題 \newline
      5.*) 關於各種~shell 的問題 \newline
      6.*) 各式各樣的~Unix \newline
      7.*) An comparison of configuration management systems (RCS, SCCS).
\end{quote}
這篇文章回答了以下問題:
\tableofcontents

因為這些都是正當合理的問題, 所以在 \url{comp.unix.questions} 或是 
\url{comp.unix.shell} 中。每隔一陣子, 就會有這些問題與答案出現, 緊接著就會
有人對同樣問題一再出現發牢騷。關於~UNIX 代表啥呢? 請參考每月 post 在 
\url{news.announce.newusers} 中名為~"Answers to Frequently Asked Questions" 
的文章。

因為~Unix 有太多不同的種類了, 所以很難保證此文件所提供的答案必然會有
用。在嘗試本文件提供的作法前, 請先讀讀你所使用系統的手冊。若你對答案
有任何建議或更正, 請送~email 給 tmtaimar@isgtec.com.

\setcounter{section}{3}
\section{自以為已經知道所有答案的人可能會問的高級問題}
\subsection{要如何在使用者不必按~RETURN 的情況下從~terminal 讀進東西?}

	在~BSD 中用~cbreak 模式,在~SysV 中則用~~ICANON 模式。

	如果你懶得用~"ioctl(2)" 來設定~terminal 的參數,也可以用~stty 來做,
        不過有點慢又沒有效率就是了。底下的程式自己看著辦吧:
\small
\begin{verbatim}
	      #include <stdio.h>
	      main()
	      {
        	    int c;

	            printf("Hit any character to continue\n");
        	    /*
	             * ioctl() would be better here; only lazy
        	     * programmers do it this way:
	             */
        	    system("/bin/stty cbreak");        /* or "stty raw" */
	            c = getchar();
	            system("/bin/stty -cbreak");
	            printf("Thank you for typing %c.\n", c);

	            exit(0);
	      }
\end{verbatim}
\normalsize
	有好幾個人送給我更正確的解法。不過很抱歉我不想把它們加進去,因為這已經
	超出這份文件的範圍了。

	通常對這個問題有興趣的人,都是想要做一些控制螢幕顯示之類的事情。如果你
	也是的話,那請參考~"curses" 的相關文件。~"curses" 是一個~portable 的螢
        幕控制函數庫。

\subsection{我要如何在未曾真的讀進東西的情況下檢查是否有字元等待讀取?}

	某些版本的~UNIX 提供了檢查某個~file descriptor 目前是否有東西待讀取的
        方法。在~BSD 中,可以用~"select(2),也可以用~FIONREAD ioctl,檢查有幾
        個字元等待讀取,不過這只對~terminal, pipe, 與~socket 有用。在~System 
        V Release 3 中可以用~poll(2),不過只對~stream 有用。在~Xenix 與~Sys V 
        r3.2 及其以後的版本裡,有一個名叫~rdchk() 的~system call 可以用來檢查
        對一個~file descriptor 做~read() 會不會卡住。

	沒有方法可以用來判斷是否有字元在~FILE pointer 中待讀取。(你可以直接查
	看~stdio 的資料結構,看看是否~input buffer 是空的,但是這方法有時會失
        效,因為你沒有辦法知道當你下一次要填滿這個~buffer 時會發生什麼事。)

	有時人們問這個問題是因為想寫
\begin{verbatim}
		if (characters available from fd)
                    read(fd, buf, sizeof buf);
\end{verbatim}
	以達成~nonblocking read。這不是一種好的做法,因為可能測的時候有東西,
        要讀的時候,已經沒有東西可讀了。正確的做法應該是用~fcntl(2) 裡的 
       ~F\_SETFL 設定~O\_NDELAY。在一些比較舊的系統~(例如: Version 7, 4.1 BSD)
	裡,並沒有 ~O\_NDELAY 可用,就得用~alarm(2) 來設定~read 的~timeout,以達成近似 
       ~nonblocking read 的功能。

\subsection{要怎樣才能得知一個已~open 檔案之檔名?}

	這個是非常困難的。若是這個~file descriptor 是對應到~pipe 或~pty 就沒
        有名字了。這個~file descriptor 對應的檔案也有可能已被刪除。若是有 
        ~symbolic link 或~hard link,則可能有許多個名字。
 
	如果你經過一再考慮後別無選擇一定要這麼做的話,可以用~find 的~-inum 與 
	~-xdev 選項,或用~ncheck,或用自己寫類似的程式來做。在這麼做時要耐心的
	等,因為在一個幾百~megabyte 甚至幾~gigabyte 的~file system中找一個檔
        案,一定得花不少時間。

\subsection{一個執行中的程式如何知道自己的~pathname?}

	若果~argv[0] 是以~"/" 開始的字,它可能就是你的程式所在地的絕對路徑。
        如果不是那就得照順序檢查~PATH 裡的每一個目錄看看裡面是否有與~argv[0]
        一樣的程式。如果找得到的話將那個目錄與程式名稱兜起來可能就是你要的
	~pathname 了。

	不過,用上述方法找到的名字並不一定是對的,因為在程式中用到
	~exec() 時, 
        ~argv[0] 是可以隨便給的。將~argv[0] 設為與要執行的程式名稱相同只
	是一種慣用法罷了!

	以下的例子可能會使你更清楚些:
\small
\begin{verbatim} 
	        #include <stdio.h>
        	main()
	        {
        	    execl("/usr/games/rogue", "vi Thesis", (char *)NULL);
	        }
\end{verbatim} 
\normalsize
	這個被執行的程式就會認為它的名字(argv[0] 之值)是 "vi Thesis")。

\subsection{如何用~popen() 對一個~process 做讀寫的動作?}

	用~pipe 把一個~process 的輸出、輸入轉給任意的~process 所可能會發生的
        問題就是~deadlock,譬如這兩個~processes 剛好同時都在等待「尚未產生」
        的輸入時。唯一能避免~deadlock 的方法就是在~pipe 的兩端都要遵循嚴格的
	~deadlock-free 協定,但是需要這些~processes 之間的互相合作才能達成,
        而對於像~popen() 這類的函數來說並不太適合。

	在~'expect' 這個軟體中附有一個能夠讓~C 程式直接引用的函式庫。其中有
        一個函式不管是在讀或寫都能達到和~popen 相同的功能。但是這個函式是使
        用~ptys 而不是~pipes,也沒有~deadlock 的問題,並且在~BSD 或~SV 中都
        能使用。若想對~'expect' 有進一步的瞭解,可參考下一個問題的解答。

\subsection{在~C 程式中要怎麼用~sleep() 才能夠~sleep 小於一秒?}

	首先要注意的是,你只能指定~delay 的「最短」時間;實際上會~delay 多久和
	系統的~scheduling 方式有關,例如系統當時的負載。如果你倒楣的話,它還可
	能會~delay 蠻長的時間。

	並沒有一個標準函式能夠在「小睡」(很短的~sleep)期間提供你計數的功能。
	某些系統有提供~usleep(n) 的函式,它能夠暫停執行 n 微秒~(microsecond)
        的時間。如果你所使用的系統沒有提供~usleep() 函式,那麼以下有可在~BSD,
	~System V 使用中的作法。

	接下來的這段程式碼是~Doug Gwyn 在~System V 中模擬~4BSD 並利用~4BSD 
	中的~select() 系統呼叫。Doung 自己都叫它為~'nap()' ;你也可以把它叫做 
	~"usleep()";
\footnotesize
\begin{verbatim}
	/*
            usleep -- support routine for 4.2BSD system call emulations
            last edit:  29-Oct-1984     D A Gwyn
	*/

      extern int        select();

      int
      usleep( usec )                            /* returns 0 if ok, else -1 */
       	    long                usec;           /* delay in microseconds */
            {
       	    static struct                       /* `timeval' */
               	    {
                    long        tv_sec;         /* seconds */
       	            long        tv_usec;        /* microsecs */
               	    }   delay;          /* _select() timeout */

            delay.tv_sec = usec / 1000000L;
       	    delay.tv_usec = usec % 1000000L;

            return select( 0, (long *)0, (long *)0, (long *)0, &delay );
       	    }

      On System V you might do it this way:

      /*
      subseconds sleeps for System V - or anything that has poll()
      Don Libes, 4/1/1991

      The BSD analog to this function is defined in terms of
      microseconds while poll() is defined in terms of milliseconds.
      For compatibility, this function provides accuracy "over the long
      run" by truncating actual requests to milliseconds and
      accumulating microseconds across calls with the idea that you are
      probably calling it in a tight loop, and that over the long run,
      the error will even out.

      If you aren't calling it in a tight loop, then you almost
      certainly aren't making microsecond-resolution requests anyway,
      in which case you don't care about microseconds.  And if you did,
      you wouldn't be using UNIX anyway because random system
      indigestion (i.e., scheduling) can make mincemeat out of any
      timing code.

      Returns 0 if successful timeout, -1 if unsuccessful.

      */

      #include <poll.h>

      int
      usleep(usec)
      unsigned int usec;                /* microseconds */
      {
            static subtotal = 0;        /* microseconds */
            int msec;                   /* milliseconds */

            /* 'foo' is only here because some versions of 5.3 have
             * a bug where the first argument to poll() is checked
             * for a valid memory address even if the second argument is 0.
             */
            struct pollfd foo;

            subtotal += usec;
            /* if less then 1 msec request, do nothing but remember it */
            if (subtotal < 1000) return(0);
            msec = subtotal/1000;
            subtotal = subtotal%1000;
            return poll(&foo,(unsigned long)0,msec);
      }
\end{verbatim}
\normalsize
	在~System V 或其他 非-BSD 的~Unix 中要使用這類的「小睡」程式,可以用 
       ~Jon Zeeff 發表在~comp.sources.misc, volume 4 中的~s5nap, 它需要安裝
	一個驅動程式,但是裝好後就可以跑得很好。(它的精確度會受到~kernel 中 
	~HZ 這個變數的影響,因為它是用到了~kernel 中的~delay() 函式。)

	現在很多較新版本的~Unix 都有提供這類的「小睡」功能了。

\subsection{如何讓~setuid 的~shell script 可以使用?}

	[ 這個問題的回答很長,但是這是一個複雜又常問的問題。在此要謝謝~Maarten 
	Litmaath 所提供的答案和以下所提到的~"indir" 程式。]

	先假設你所用的~UNIX 是能認得「可執行的~shell script」的變異過的~UNIX
       (如 4.3BSD 或 SunOS)。這類~script 的第一行一定是如以下一般:
\begin{verbatim}
		#!/bin/sh
\end{verbatim}
	這樣的~script 就是所謂可執行的~script,因為它和一般可執行的~binary 檔
        一樣有~magic number 做開頭。在我們所用的例子中,~magic number 為 
        '\#!',~OS 會把這行接下來的部份當作這整個~script 的解譯程式,其後可能還
        會有一些~option 如:
\begin{verbatim}
		#!/bin/sed -f
\end{verbatim}
	假設這個 script 的名字叫做 'foo',並且放在 /bin 下,那麼如果你用:
\begin{verbatim}
		foo arg1 arg2 arg3
\end{verbatim}
	那麼 OS 實際在執行時會把它看成是:
\begin{verbatim}
		/bin/sed -f foo arg1 arg2 arg3
\end{verbatim}
	有一點不同的是:如果 'foo' 設定成 setuid,那麼 OS 會把它以第一種格
        式來解釋;如果你硬是以第二種格式輸入,那麼 OS 會以 /bin/sed 的 
        permission 為準,而它當然不會是 setuid。

        ----------

	好吧,那如果我的 shell script 並不是以~'\#!' 做開頭,
	或是我的~OS 根本就不認得它呢?

	嗯,如果這個~shell(或是其他的解譯程式)試著要去執行它,那麼~OS 會傳回
	一個錯誤訊息,表示這個檔案不是以合法的~magic number 做開頭。收到這個錯
	誤訊息後,~shell 會把這個檔案認定成是~shell script,並以另一種方式來執行
\begin{verbatim}
		/bin/sh shell_script arguments
\end{verbatim}
	但是我們在前面已經看到了,在這樣的情形下,設成為~setuid 的 
        ~shell\_script 並不會發生作用。

        ----------

	那麼,設成~setuid 的~shell script 到底有什麼安全上的問題呢?

	嗯,假設這個~script 叫做~'/etc/shell\_script',它的開頭是:
\begin{verbatim}
		#!/bin/sh
\end{verbatim}
	現在我們來看看以下的命令會發生什麼事:
\begin{verbatim}
		$ cd /tmp
	        $ ln /etc/setuid_script -i
        	$ PATH=.
	        $ -i
\end{verbatim}
	我們可以看出來,以上的最後一個命令會被解釋成:
\begin{verbatim}
		#!/bin/sh -i
\end{verbatim}
	而這樣的命令會讓我們得到一個可以輸入命令的 shell,並且會被 setuid 成
        這個  script 的擁有者。
	幸好,這樣的安全漏洞可以很輕易地防止,只需要把第一行改成:
\begin{verbatim}
		#!/bin/sh -
\end{verbatim}
	'-' 這個符號代表著它是整個~option list 的結尾:所以如果再用前述的方法
        的話,'-i' 就會如本來所期望的被解釋成~script 檔案的名字。

        ----------

	然而,還有更嚴重的問題:
\begin{verbatim}
		$ cd /tmp
	        $ ln /etc/setuid_script temp
        	$ nice -20 temp &
	        $ mv my_script temp
\end{verbatim}
	第三個命令會被解釋成:
\begin{verbatim}
		nice -20 /bin/sh - temp
\end{verbatim}
	而因為這個命令的優先權被設得很低,那麼第四個命令可能就有機會搶先在 
	~shell 開啟~'temp' 之前就用~'my\_script' 把~'temp' 給蓋掉!有四種方法
        可以修補這個安全上的漏洞:

	1)  讓~OS 用另一個比較安全的方式執行~setuid script。如~System V R4 和 
	   ~4.4BSD 利用~/dev/fd 來把該~script 的~file descriptor 傳給解譯程式。

	2)  透過一個前端程式來間接解譯要執行的~script,以確定在真正的解譯程式
            啟動前一切正常。例如,你可以用~comp.sources.unix 中的~'indir' 程
            式,那麼你的~script 開頭就會像這樣:
\begin{verbatim}
			#!/bin/indir -u
	                #?/bin/sh /etc/setuid_script
\end{verbatim}
	3)  造一個~'binary wrapper':一個真正的~setuid 可執行程式,這個程式的
            唯一功能就是用來執行~script 中所指定的解譯程式,並以該~script 的檔
            名為參數傳給解譯程式。

	4)  造一個~'setuid script server' ,並把一些要用到、檢查過的~setuid 
           ~script 存放在~database 中。當成功地被呼叫後,會去執行正確的解譯程
            式及正確的~script。

        ---------

	現在我們已經能確定所解譯到的~script 是正確的,那麼還有其他的危險嗎?

	很抱歉,當然還有!在使用~shell scipt 的時候,你一定不能忘記要把~PATH 
        這個變數很明確地設到正確的路徑去。你能夠指出這是為什麼嗎?除此之外,
        還有~IFS 這個變數如果沒設好也可能會造成麻煩。其他的環境變數也可能會造
        成安全上的問題,如~SHELL... 更重要的,你必須要確定在 script 中沒有命
        令會讓它產生出可下命令的~shell(interactive shell escape)!還有就是,
       ~umask 可能被設成奇怪的值等等...

	除此之外,你應該要知道~setuid script 會「繼承」所有它所用到的命令的 
       ~bug 及安全問題。

	總而言之,你應該知道~setuid shell script 真的是件非常危險的事吧!
        最好還是寫~C 程式啦。

\subsection{我要如何得知有哪些~process 開了某一檔案,或某一~process 正在使用
      哪一個~fileystem(以至於我無法~unmount 這個~filesystem)?}

	你可以用~fuser(system V),~fstat(BSD),~ofiles(public domain) 或是
	~pff (public domain)。這些程式可以告訴你哪些~processes 正在使用哪些
	檔案。

	4.3BSD 的~fstat,有一份~Dynix,~SunOS 與~Ultrix 都可以用的~port。
	你可以找找放 comp.sources.unix, volume 18的地方。

	Pff 是~kstuff 這套軟體的一部分,很多系統上都可以用。欲取得~kstuff 請參
        考問題~3.10。

\subsection{我要怎麼知道是誰在~finger 我啊?}

	一般來說,你是無法找出在遠端機器 finger 你的那個人的 userid 的。你大概
        只能找出在從哪台機器 finger 的。另外有一種可行的方法,如果你的系統支援
        並且假設 finger daemon 不反對的話,那麼可以把你的 .plan 檔用 "named 
        pipe" 而不用一般的文字檔。(用 'mknod' 來造)

	接下來,你執行一個程式去寫(open for writing)你的 .plan 檔;但是由於
        你的 .plan 是一個 "named pipe",所以這個開檔的動作要一直等到有其他的 
	process 去讀(open for reading)你的 .plan 檔時才會成功。現在你就可以
        任意地把你所想讓人 finger 到的 .plan 內容寫入這個 pipe。在 
        comp.sources.misc, volumn 41 中有個 "planner" 的程式可以做這件事。

	當然,如果你的系統不支援 "named pipe",或是你所用的 finger 程式只接受
        純文字的 .plan 檔,那麼以上的方法就行不通了。

	你的程式也可以藉由查看 "netstat" 的輸出,來找出這次的 finger 是從那裡
        連過來的,但是這並無法看出遠端執行 finger 的人是誰。

	想要知道遠端的人是誰,必須要遠端的機器有跑支援如 RFC 931 的識別程式才
	行。

	現在,在常見的~BSD 系統上就有三種~RFC 931 的實做程式,同時也有許多的
	應用程式(如~wuarchive ftpd)支援。如果你想得到更多有關~RFC 931 的資
        訊,可以加入~rfc931-user 的~mailing list,
        \url{rfc931-users-request@kramden.acf.nyu.edu}。

	另外還有三個注意事項。第一,有許多的 NFS 系統無法正確地處理 "named 
	pipe"。這個意思是,當你要去讀放在另一部機器上的 pipe 時,可能會被系統 
	block 住直到time out,或是因為 pipe 的檔案長度是 0 所以系統就不把它的
        內容列出來。

	第二,在許多的系統中,fingerd 會在讀 .plan 檔之前先去檢查它裡面是否真
        的有資料並且是可以讀的。這樣就會造成遠端 finger 你的人根本就看不到你
        的 .plan 檔,因為你的 .plan 長度是 0。

	第三,支援 "named pipe" 的系統通常在同一個時間裡只允許系統中存在某個固
	定數目的 pipes,檢查 kernel 的 config 檔和 FIFOCNT 選項即可得知。如果
        系統中的 pipe 數目超過了 FIFOCNT,那麼系統就會暫停所有新的 pipe 直到有
        人關掉他先前所開的 pipe。這是因為 pipe 所用到的 buffer 是放在 
        non-paged 的記憶體中。

\subsection{能不能在一個~process 和~terminal 的連接已經斷掉之後再接回來,例如
       在~background 跑一個程式然後就~logout 而斷掉的程式?}

	大部份版本的~Unix 都不像~VMS 和~Multics 等作業系統支援~"detaching" 和 
	~"attaching" process。不過,有兩個免費的軟體能夠幫你達成這個目的。

	第一個是 "screen",在 comp.sources.unix 中的標題是 "Screen, multiple 
        windows on a CRT"。(你可以在 comp.sources.misc, volumn 28 中發表的 
        "screen-3.2" 中找到)這個軟體在 BSD,System V r3.2 及 SCO UNIX 皆可執
        行。

	第二個是 "pty",在 comp.sources.unix 中的標題是 "Run a program under 
        a pty session"。(可以在 volumn 23 發表的 "pty" 中找到)"pty" 只能在 
        BSD 的系統中執行。

	以上這兩個軟體都沒有溯及既往的能力,也就是說,你如果想對某個~process
        做~detach 或~attach 的動作,那麼就必須要先在~screen 或~pty 下啟動它才
        行。

\subsection{有沒有辦法可以偷聽一個~terminal,就是說將其輸出複製一份至其他的 
     ~terminal。}

	有幾種方法可以達成這個目的,不過沒有一個是完美的:
\begin{itemize}
	\item  kibitz 允許兩個(或更多)的人透過~shell(或其他的程式)彼此溝通。
           它的用途有:
	   \begin{itemize}
		\item  監視或援助令一人的~terminal session;
		\item  記錄所有的輸出入以用來能捲頁回去,儲存整個輸出入,甚至是可
                   以直接編輯它。
		\item  當團隊合作時,如製作 game、撰寫文件等工作,每個人都各有所長
                   各有所短,就可用此種方式彼此互補。
	   \end{itemize}
	kibitz 是~'expect' 軟體的一部份,請查看問題~3.9。

	kibitz 需要有被監視人的同意才能進行。如果想不取得同意就進行監視,那就
        得用一些比較令人討厭的方法了:

	\item  你可以自己寫一個程式去搜查整個~kernel 的結構,監視~terminal 所
	   用的~output buffer,然後把它所輸出的字全抓下來。很明顯的,這是熟悉
	   ~Unix kernel 的人才可能做得到的。但是,不論你是用什麼方法大概都無法
           拿到其他的~Unix 去用。

	\item 如果你是想要隨時監視一個特定、實際拉線連上的~terminal(例如,你想讓
	   管理者能夠從其他機器的終端機查看某部機器的~console),那你可以乾脆
           接一台監視器在通往那台~terminal 的線上。舉例來說,把你的監視器的輸
           出接到另一台機器的~serial port,然後執行一個程式去收集那個 port 的
           輸入並把它轉到另一個~port 去,而這個~port 就真的連到你所要監視的 
          ~terminal 去。這樣做的話,你必須確定從該~terminal 來的輸出會隨著線傳
           回去,即使你只是插接在電腦和~terminal 的中間,這並不會太難做到。 用
           這個方法,對~terminal 佈線不太熟悉的人是無法發覺的。

	\item  在最近一版的~screen 中有一種~multi-user 模式可以用。若想多了解一
           點~screen,請看問題 4.10。

	\item  如果你所使用的系統有提供~stream(如~SunOS,SVR4)那麼你可以使用
	   發表在~comp.sources.misc, volumn 28 中的~advise 程式。它並不需要先
           執行。(但是,你必須要事先把你的系統設定成在開啟~tty 或~pty 時會自
           動把~advise 模組放入~stream 中。)
\end{itemize}

\end{CJK*}
\end{document}
