\documentclass{article}
\usepackage{CJK}
\usepackage{url}

\begin{document}
\begin{CJK*}{Bg5}{ming} 
這七篇文章包含一些在~comp.unix.questions 和~comp.unix.shell 常見到的問
題。請不要再問這些問題,因為這些問題已經回答過太多次了。但也請不要因為
有人問這些問題而發火,因為他們可能尚未讀過這些文章。

This collection of documents is Copyright (c) 1994, Ted Timar, except 
Part 6, which is Copyright (c) 1994, Pierre Lewis and Ted Timar.

All rights reserved. Permission to distribute the collection is 
hereby granted providing that distribution is electronic, no money is 
involved, reasonable attempts are made to use the latest version and 
all credits and this copyright notice are maintained.

Other requests for distribution will be considered.

All reasonable requests will be granted.

中文翻譯~by \url{{chenjl,freedom,jjyang}@csie.nctu.edu.tw}
若您對中文翻譯有任何意見請發~e-mail 給~\url{cfaq@csie.nctu.edu.tw}

我們希望這些文件中的資訊能對你有所幫助﹐但是並不保証是正確的。若發生損
害請自行負責。

您可以在~rtfm.mit.edu 的~pub/usenet/news.answers 找到包括此文件在內的
許多~FAQ。 在此目錄下的~FAQ 的名字可在文章的頂端的~"Archive-Name:" 
那一行找到。\footnote{ 在台灣請用~\url{ftp://NCTUCCA.edu.tw/USENET/FAQ},
若您人在交大的話 ~\url{ftp://ftp.csie.nctu.edu.tw/pub/Documents/FAQ} 是從
~CCCA mirror  來的}

此一~FAQ 是以~"unix-faq/faq/part[1-7]" 為名。

這些文章大約分成:
\begin{quote} 
      1.*)一般性的問題 \newline
      2.*)初學者可能會問的基本問題 \newline
      3.*) 中級的問題 \newline
      4.*) 自以為已經知道所有答案的人可能會問的高級問題 \newline
      5.*) 關於各種~shell 的問題 \newline
      6.*) 各式各樣的~Unix \newline
      7.*) An comparison of configuration management systems (RCS, SCCS).
\end{quote}
這篇文章回答了以下問題:
\tableofcontents

因為這些都是正當合理的問題, 所以在 \url{comp.unix.questions} 或是 
\url{comp.unix.shell} 中。每隔一陣子, 就會有這些問題與答案出現, 緊接著就會
有人對同樣問題一再出現發牢騷。關於~UNIX 代表啥呢? 請參考每月 post 在 
\url{news.announce.newusers} 中名為~"Answers to Frequently Asked Questions" 
的文章。

因為~Unix 有太多不同的種類了, 所以很難保證此文件所提供的答案必然會有
用。在嘗試本文件提供的作法前, 請先讀讀你所使用系統的手冊。若你對答案
有任何建議或更正, 請送~email 給 tmtaimar@isgtec.com.

\setcounter{section}{5}
\section{各式各樣的 Unix}
\subsection{聲明,介紹及感謝。}

    我並無法保證以下內容的完整性及正確性。我只是利用有限的時間盡量去做
    (常碰到互相衝突的資料),未來還有很多要做。我希望能持續改進這份文
    件。歡迎您的批評與指教:lew@bnr.ca。
    
    首先讓我們先為 UNIX 下個簡短的定義。我們所提的 Unix 指的是一個通
    常是由 C 寫成的作業系統,它有階層式的檔案系統,統合了檔案和裝置(device)
    I/O,其系統函式呼叫(system call)介面包含了如 fork(),pipe() 等服務,
    而且它的使用者介面包含 cc,troff,grep,awk 之類的工具及某一種
    shell。UNIX 過去是 USL(AT\&T) 的註冊商標,現在則是 X/Open 的註冊
    商標。本文所指的 UNIX 是一般通用的意義,而不是那個註冊商標。
    
    絕大多數的 Unix 或多或少都用了來自 AT\&T(現在則是 Novell)的程式
    碼(大部分的 Unix 裡面可能都還有使用第一版 C 語言所寫的程式),
    但是也有些是自己從頭寫一個一模一樣的 Unix。(就是寫得和 Unix 完全
    相容但是卻沒有用到 AT\&T 的程式碼。)
    
    此外還有一些建構於別種 OS 上的 Unix-like 環境,例如 VOS;以及向 UNIX
    借用靈感的的 OS,例如 MS-DOS。這些都不在本文的討論範圍裡。對即時(
    real-time)的 Unix 本文也討論得不多。
    
    UNIX 的流派實在多得令人難以置信。主要的原因是因為 Unix 的原始程式容易
    取得、修改與移植。一般廠商的典型作法是以某一流派為主體再加入其他流
    派的特色。如此一來又產生了另一個新的流派。目前 Unix 有數百款,如果
    說有一種可當成圭臬的話,那應該是 System V 吧。
    
    本文的內容大部份取自於網路上流傳的資訊。如果取自其他來源,則會在適當的章
    節中加以說明。
   
    \urlstyle{rm} 
    \newcommand\email{\begingroup \urlstyle{rm}\Url}
    \urldef{\myself}{\email}{root%candle.uucp@ls.com}
    特別感謝:
    ~\url{pat@bnr.ca,guy@auspex.com}, \url{pen@lysator.liu.se}, 
    \url{mikes@ingres.com}, \url{mjd@saul.cis.upenn.edu}, 
    \myself, 
    \url{ee@atbull.bull.co.at}, \url{Aaron_Dailey@stortek.com}, 
    \url{ralph@dci.pinetree.org},
    \url{sbdah@mcshh.hanse.de}, \url{macmach@andrew.cmu.edu}, 
    \url{jwa@alw.nih.gov} [4.4BSD],
    \url{roeber@axpvms.cern.ch}, \url{bob@pta.pyramid.com.au},
    \url{bad@flatlin.ka.sub.org}, 
    \url{m5@vail.tivoli.com}, \url{dan@fch.wimsey.bc.ca}, 
    \url{jlbrand@uswnvg.com},
    \url{jpazer@usl.com},
    \url{ym@satelnet.org},
    \url{merritt@gendev.slc.paramax.com},
    \url{quinlan@ygg.drasil.com},
    \url{steve@rudolph.ssd.csd.harris.com},
    \url{bud@heinous.isca.uiowa.edu}, \url{pcu@umich.edu}, 
    \url{quinlan@yggdrasil.com},
    \url{Dan_Menchaca@quickmail.apple.com},
    \url{D.Lamptey@sheffield.ac.uk}, 
    \url{derekn@vw.ece.cmu.edu}, \url{gordon@PowerOpen.org}, 
    還有許多我忘了名字的人,以及許多我曾拜讀他們文章的人。


\subsection{Unix 簡史}

    Unix 的歷史開始於~1969,Ken Thompson、Dennis Ritchie (K\&R 裡的那個 
    R) 與一群人在一部『位於角落且乏人問津的~PDP-7』上進行的一些工
    作,後來這個系統變成了~Unix。"UNIX" 這個字(最初是寫成~Unics, 
    Uniplexed Information and Computering System)有一點玩弄~"Multics" 這個字
    的意味。

    最初十年間,Unix 的發展基本上都是在~Bell Labs 裡完成的。最初的幾個
    版本稱為~"Version n" 或~"Nth Edition" ,是給~DEC 的~PDP-11(16 bits) 與
    其下一代產品~Vax(32 bits)用的。主要的幾個版本為:
\begin{itemize}
    \item V1 (1971):  第一版的 Unix,以 PDP-11/20 的組合語言寫成。包括檔案系統
        (file system)、fork()、roff、ed 等東西。是用來給 AT\&T 的專利部門
        處理文件用的。Pipe() 出現於 V2。
      
    \item V4 (1973):  以 C 語言從頭寫過,這或許是 OS 歷史上最重要的一個事
        件,這表示 Unix 修改容易,可以幾個月內移植到新的硬體架構上。最
        初 C 語言是為 Unix 設計的,所以 C 與 Unix 間有緊密的關係。

    \item V6 (1975):  第一個在 Bell Labs 外(尤其是大學中)廣為流傳的 Unix 版
        本。這也是 Unix 歧異的起點與廣受歡迎的開始。1.xBSD(PDP-11) 就
        是由這個版本衍生出來的。J. Lions 的“A Commentary on the Unix 
        Operating System" 也是以 V6 為本。

    \item V7 (1979):  在許多~Unix 玩家的心目中,這是『最後一個真正的~UNIX』,
        也是『空前絕後的一個~Unix』[Bourne 說的],這個版本包括一個完整
        K\&R C、Bourne shell。V7 移植到~VAX 上稱為~32V。~V7 的~kernel 
        只有~40 Kbytes!
    
    底下列出 V7 的系統呼叫,供「後進之輩」們憑弔瞻仰:
	\begin{quote}
        \_exit, access, acct, alarm, brk, chdir, chmod, chown,
        chroot, close, creat, dup, dup2, exec*, exit, fork, fstat,
        ftime, getegid, geteuid, getgid, getpid, getuid, gtty,
        indir, ioctl, kill, link, lock, lseek, mknod, mount,
        mpxcall, nice, open, pause, phys, pipe, pkoff, pkon,
        profil, ptrace, read, sbrk, setgid, setuid, signal, stat,
        stime, stty, sync, tell, time, times, umask, umount,
        unlink, utime, wait, write.
	\end{quote}
\end{itemize}
    以上這些~"Vn" 版都是由~Bell Labs 裡的~Computer Research Group(CRG) 
    發展的。另一個~Unix Support Group(USG) 負責支援服務。Bell Labs 裡還
    有另一個與~Unix 發展相關的團體~Programmer's WorkBench(PWB) 則做出
    了~sccs、named pipe 及一些其它的東西。USG 與~PWB 後來於 1983 年
    合併成~Unix System Developement Lab。

    此外~Bell Labs 在~Columbus 的分支機構,負責發展~Operations Support 
    System 也做了一版的~Unix 稱為~CB Unix (Columbus Unix)。System V IPC 
    就是從~CB Unix 來的。

    到了1980 年代~Bell labs 並未放棄~Unix。CRG 仍繼續發展~V 系列的
    ~Unix (Stroustrrup 在它的 C++ 第二版裡就提到了~V10),不過並未對外發
    表。目前發展 Unix(System V) 的公司是~Unix System Laboratories(USL)。
    USL 本為~AT\&T 所有,'93 年初被~Novell 收購。Novell 於 '93 年末將
    ~"UNIX" 這個註冊商標轉給~X/Open。

    除了 AT\&T 所屬的機構外,有不少地方也對 Unix 的改進做出了貢獻,例
    如 Berkeley 就自成一大流派。有不少廠商(尤其是賣工作站的)也對 Unix 
    的發展有所貢獻(如 Sun 的 NFS)。
    
    對任何對 Unix 有興趣的人而言 Don Libes 與 Sandy Ressler 所寫的 "Life
    with Unix" 是一本有趣的書。
    \footnote{Life with Unix 在台灣並沒有書局代理進口, 非常可惜!
       這本書裡面有很多關於 Unix 的故事, 有興趣可以來我們這兒泡泡茶,
       看看這本故事書。} 此書講了許多 Unix 的歷史與發展及一些軼
    事。本文多採此書之說。

\subsection{主要的~Unix 流派}

    目前為止,UNIX 有兩大流派:那就是 AT\&T 的 System V (讀 five, 不讀 v)
    與 BSD (Berkeley Software Distribution)。SVR4 是兩大流派融合後的產物。
    '91 年底,與 System V 針鋒相對的 Open Software Foundation 推出了 OSF/1,
    或許 OSF/1 會改變市場生態。
    \footnote{由今日(\today)觀之,OSF/1 應該是無法挑戰 System V 了}

    以下是~System V、BSD、OSF/1 的主要版本以及特色。
    
    AT\&T 的 System V。Intel 系列的機器多半使用此系列。移植版本最多的 Unix,
    當然在移植時都會加入一些由 BSD 發展的有用功能,例如 csh、job control、
    termcap、curses、vi、symbolic link。目前 System V 的發展是由 Unix
    International(UI) 負責監控。UI 的成員包括 AT\&T、Sun 等。
    網路討論區: comp.unix.sysv[23]86。主要版本:
\begin{itemize}
         \item  System III (1982): AT\&T 第一個拿來賣錢的 UNIX
	 \begin{itemize}
           \item FIFOs (named pipes)  (later?)
	 \end{itemize}

         \item System V (1983):
	 \begin{itemize}
           \item IPC package (shm, msg, sem)
	 \end{itemize}

         \item SVR2 (1984):
	 \begin{itemize}
           \item shell 函數 (sh)
           \item SVID (System V Interface Definition)
	 \end{itemize}

         \item SVR3 (1986) for ? platforms:
	 \begin{itemize}
           \item STREAMS (從 V8 得來的靈感), poll(), TLI (網路軟體)
           \item RFS
           \item 共用程式庫(shared libs)
           \item SVID 2
           \item demand paging (如果硬體有支援的話)
	 \end{itemize}

         \item SVR3.2:
	 \begin{itemize}
           \item 併入 Xenix (Intel 80386)
           \item 網路
	 \end{itemize}

         \item SVR4 (1988), 融合了 System V、BSD、SunOS 是各種 UNIX 中 
           的主流
	 \begin{itemize}
           \item 取自 SVR3 者: 系統管理, terminal 界面, 印表機 (from BSD?),
             RFS, STREAMS, uucp
           \item 取自 BSD 者: FFS, TCP/IP, sockets, select(), csh
           \item 取自 SunOS 者: NFS, OpenLook GUI, X11/NeWS,
             具有記憶體映對檔案的虛擬記憶體子系統(virtual memory
             subsystem with memory-mapped files), 共用程式庫
             (!= SVR3 ones?)
           \item ksh
           \item ANSI C
           \item 國際化(Internationalization) (8-bit clean)
           \item ABI (Application Binary Interface -- routines instead of traps)
           \item POSIX, X/Open, SVID3
	 \end{itemize}

         \item SVR4.1
	 \begin{itemize}
           \item 非同步 I/O (from SunOS?)
	 \end{itemize}

         \item SVR4.2 (based on SVR4.1ES)
	 \begin{itemize}
           \item Veritas FS, ACLs
           \item 動態載入核心模組
	 \end{itemize}

         \item Future:
	 \begin{itemize}
           \item SVR4 MP (多處理器)
           \item 使用 Chorus 微核心?
	 \end{itemize}
\end{itemize}

    Berkeley Software Distribution (BSD)。VAX、RISC、各式工作站多用之。
    比起 System V 來 BSD 的變動比較快而且學術研究的味道比較濃一點。
    Unix 之所以能夠流行,BSD 居功闕偉。許多對 Unix 的加強改進都是由
    BSD 先做出來的。在 UCB (University of California at Berkeley) 中負責 BSD 
    的是 Computer System Research Group(CSRG)。CSRG 已於 1992 年關門大吉。
    網路討論區: comp.unix.bsd。主要的版本如下:
\begin{itemize}
         \item 2.xBSD (1978) 給 PDP-11 用的, 這個系統好像還活著的樣子(1992 
           還推出了 2.11BSD!).
	   \begin{itemize}
           \item csh
	   \end{itemize}

         \item 3BSD (1978):
	   \begin{itemize}
           \item 虛擬記憶體
	   \end{itemize}

         \item 4.?BSD:
	   \begin{itemize}
           \item termcap, curses
           \item vi
	   \end{itemize}

         \item 4.0BSD (1980):

         \item 4.1BSD (?): 後來 AT\&T CRG 版本皆以此為本
	   \begin{itemize}
           \item job 控制
           \item automatic kernel config
           \item vfork()
	   \end{itemize}

         \item 4.2BSD (1983):
	   \begin{itemize}
           \item TCP/IP, sockets, ethernet
           \item UFS: 長檔名, symbolic links
           \item 新的 reliable signals (SVR3 採用了 4.1 的 reliable signals)
           \item select()
	   \end{itemize}

         \item 4.3BSD (1986) for VAX, ?:
         \item 4.3 Tahoe (1988): 4.3BSD 附加對 Tahoe(一款32位元的超級迷你電腦)
           的支援及一些新東西
	   \begin{itemize}
           \item Fat FFS
           \item 新的 TCP 演算法
	   \end{itemize}
         \item 4.3 Reno (1990) for VAX, Tahoe, HP 9000/300:
	   \begin{itemize}
           \item 大部份的 P1003.1
           \item NFS (from Sun)
           \item MFS (記憶體檔案系統)
           \item OSI: TP4, CLNP, ISODE's FTAM, VT and X.500;  SLIP
           \item Kerberos
	   \end{itemize}

         \item Net1 (?) 與 Net2 (June 1991) 磁帶: BSD 中不侵犯 USL 版權的部份

         \item 4.4BSD (alpha June 1992) for HP 9000/300, Sparc, 386, DEC, others;
           已經不支援 VAX 與 Tahoe; 有兩個版本, 一個是 lite (大約是 Net2 的
           內容,加上修正與新的架構); 另一個是 encumbered (內容一應俱全,但需
           USL 授權):
	   \begin{itemize}
           \item 以 Mach 2.5 為基礎的新型虛擬記憶體系統 (VMS),
           \item 虛擬檔案系統介面, log-structured 檔案系統, 本地檔案系統
             的大小可達 $2^63$, NFS (可以免費流傳,可以跟 Sun 的 NFS 並存,
             架在 UDP 或 TCP 之上)
           \item 支援 ISO/OSI 網路架構~(以~ISODE 此一~implementation 為基礎): TP4/CLNP/802.3 以及
             ~TP0/CONS/X.25, session 及更高層的協定則放在~user space;
             FTAM, VT, X.500。
	     \footnote{FTAM 跟~TCP/IP 網路的~ftp 協定相當,
                      ~VT 則跟~telnet 相當,~X.500 則是~directory service}

           \item 大部分是 POSIX.1 (特別是新增的 SV 形式終端機驅動程式),有很多
             POSIX.2, 改進過的 job control; ANSI C 表頭檔
           \item Kerberos 以整合入系統內許多地方 (包含 NFS)
           \item TCP/IP 加強 (包含表頭預測, SLIP)
           \item 重要的核心修改 (新式系統呼叫慣例, ...)
           \item 其他改進: FIFOs, 以位元組為範圍做檔案鎖定
	   \end{itemize}
           正式的 4.4BSD 版本原來預計在 alpha 版 6 個月後發表。
           \footnote{結果是在 '93 年六七月間發表}
\end{itemize}

    Open Software Foundation(OSF) 於~1991 年底推出了~OSF/1。OSF/1 需要 
    ~SVR2 授權。符合~SVID 2、~SVID 3、~POSIX、~X/Open 等標準。
    ~Apollo, Dec, HP, IBM 等大廠商都是~OSF 的成員。
\begin{itemize}
         \item OSF/1 (1991):
	 \begin{itemize}
           \item 以 ~Mach 2.5 的核心為基礎
           \item 對稱式多重處理, 平行化的核心, 處理緒(thread)
           \item 邏輯式容量(logical volumes), 磁碟鏡射(disk mirroring),
             UFS (原生的), S5 FS, NFS
           \item 系統安全之加強(B1 加一些 B2, B3 或 C2), 4.3BSD 的系統管理
           \item STREAMS, TLI/XTI, sockets
           \item 共用程式庫, 動態程式載入器 (包括核心)
           \item Motif GUI
	 \end{itemize}

         \item Release 1.3 (Jun 94)
	 \begin{itemize}
           \item 以 MACH 3.0 的微核心為基礎
           \item 符合目前規格 1170 草案的標准
            (在 X/Open 的 Fast Track 程序中考慮過要將此標準化)
           \item Data Capture I/F, Common Data Link I/F,
           \item 支援ISO 10646 與 64-bit
           \item 以 Mach 3.0 為基礎的 OSF/1 MK (mircokernel)
	 \end{itemize}
\end{itemize}

    以上關於 Unix 主要流派的列表或許該把 Microsoft 的 Xenix 也列出,因
    為有不少 Unix 的分支是拿 Xenix 去改的。Xenix 是從 V7、System III、
    System V 改出來的,外觀沒什麼重大的改變,內部則為了求得在微電腦上
    使用時有較好的表現而做了不小的更動。

    關於~Unix 兩大流派的內部有兩本好書可供參考。
\begin{itemize}
      \item System V: "Design of the Unix Operating System", M.J. Bach.
      \item BSD: "Design and Implementation of the 4.3BSD Unix Operating System",
        Leffler, McKusick, Karels, Quaterman.
\end{itemize}
    關於~OSF/1 的介紹可參考~O'Reilly 出版的~"Guide to OSF/1, A Technical 
    Synopsis" 一書。關於~SunOS,可參考~Summer 1989 USENIX Proceedings
    裡的~"Virtual Memory Architecture in SunOS" 與~"Shared Libraries in 
    SunOS"。

    92 年~4 月號的~Unix Review 有一系列關於各種~Unix 之發展方向。BSD-
    FAQ極有參考價值,本文中所提到的幾個網路討論區也值得看看。

\subsection{Unix 的標準化}

     目前~(95年初) ~Unix 標準化的主要參與者:
\begin{itemize}
      \item Novell 在~93 年初買下~USL 成為原始程式的擁有者。
      \item X/Open 可決定誰能使用~"UNIX" 這個商標當產品名稱。
      \item OSF 具有雙重身分:其一為~OSF/1 與~Motif 的發展者,其二為 COSE 
        之發展的監控組織。 OSF 於~1994 年重組後,Sun 成為了~OSF 中
        的一員,OSF 與~X/Open 之間的關係也正常化了。
      \item IEEE 訂定~POSIX 與~LAN 的相關標準。
      \item IBM、Apple、Motorola、Bull 以及其他廠商合組了~PowerOpen 來推動
        PowperPC。別把它與一個也叫做~PowerOpen 的圖形環境搞混了。
\end{itemize}

     底下是一些與 Unix 有關的標準之簡述:

\begin{itemize}
      \item IEEE:
      \begin{itemize}
        \item 802.x (LAN) standards (LLC, ethernet, token ring, token bus)
        \item  POSIX (ISO 9945?): Portable Operating System I/F (Unix, VMS
          and OS/2!) (目前唯一已定案的標準?)
        \begin{itemize}
          \item 1003.1:  函數庫(大部分是 system call) -- 除了 signals 與 
		     terminal 界面外多取自於 V7
          \item 1003.2:  shell 與公用程式
          \item 1003.3:  測試方法與合格標準
          \item 1003.4:  real-time: binary semaphores, process memory
                     locking, memory-mapped files, shared memory,
                     priority scheduling, real-time signals, clocks and
                     timers, IPC message passing, synchronized I/O,
                     asynchronous I/O, real-time files
                     \footnote{翻了反而看不懂... :)
                     即時性: 雙態信號、執行體記憶體鎖定、記憶體映對檔案、
                     共用記憶體、優先序排程、即時通告、時脈與定時器、
                     IPC 訊息傳送、同步輸出入、非同步輸出入、即時檔案。}
          \item 1003.5:  Ada language bindings
          \item 1003.6:  系統安全
          \item 1003.7:  系統管理(包括印表)
          \item 1003.8:  透通式檔案存取(transparent file access)
          \item 1003.9:  FORTRAN language bindings
          \item 1003.10: 高速計算(super computing)
          \item 1003.12: 與協定種類無關的介面(protocol-independent I/Fs)
          \item 1003.13: 即時外觀(real-time profiles)
          \item 1003.15: 高速計算批次處理介面(supercomputing batch I/Fs)
          \item 1003.16: C-language bindings (?)
          \item 1003.17: directory services
          \item 1003.18: POSIX standardized profile
          \item 1003.19: FORTRAN 90 language bindings
        \end{itemize}
      \end{itemize}

      \item X/Open (由廠商籌設的組織, 成立於 1984 年):
      \begin{itemize}
        \item X/Open Portability Guides (XPGn):
	\begin{itemize}
          \item XPG2 (1987), 非常傾向 SV
            Vol 1:  命令與公用程式
            Vol 2:  系統呼叫與函數庫
            Vol 3:  terminal 界面(curses, termio), IPC (SV),
                    國際化
            Vol 4:  程式語言 (C, COBOL!)
            Vol 5:  資料管理(ISAM, SQL)
          \item XPG3 (1989) adds: X11 API
          \item XPG4 (1992) adds: XTI? 22 個元件
	\end{itemize}
        \item XOM 系列的介面:
	\begin{itemize}
          \item XOM (X/Open Object Management) 需遵循的通用介面機制(generic
            I/F mechanisms for following)
          \item XDS (X/Open Directory Service)
          \item XMH (X/Open Mail ??)
          \item XMP (X/Open Management Protocols) -- 不是 Bull's CM API?
	\end{itemize}
        \item X/Open 此時已有權管理 "UNIX" 商標 ('93 年底);
        \item "Spec 1170"
	\begin{itemize}
          \item 此規格目前正在籌備當中,是一個共通的 API, 要使用 UNIX 這個
            名稱的廠商必需遵循此 API 的規格。這是結合 SVID、OSF 的 AES
            與其他東東而成。\footnote{請參考\url{http://www.UNIX-systems.org/}}
	\end{itemize}
      \end{itemize}

      \item AT\&T
        (在 1994 年後這些已經無關緊要了? 現在是誰負責~SVID, TLI, APLI?)
	\begin{itemize}
        \item System V Interface Definition (SVID)
          \item SVID1 (1985, SVR2)
	  \begin{itemize}
            \item Vol 1:  系統呼叫與函數庫(類似XPG2.1)
	  \end{itemize}
          \item SVID2 (1986, SVR3)
	  \begin{itemize}
            \item Vol 1:  系統呼叫與函數庫(基礎,核心延伸)
            \item Vol 2:  指令與公用程式 (基礎,進階,管理,軟體發展
                    ), 終端機介面
            \item Vol 3:  終端機介面(又來了), STREAMS and TLI, RFS
	  \end{itemize}
          \item SVID3 (19??, SVR4) adds
	  \begin{itemize}
            \item Vol 4:  ??  etc
	  \end{itemize}
	\end{itemize}
	\begin{itemize}
        \item APIs
          \item Transport Library Interface (TLI)
          \item ACSE/Presentation Library Interface (APLI)
	\end{itemize}

      \item COSE (COmmon Open Software Environment) [IBM, HP, SunSoft, others]:
        目的在使不同的 Unix 平台可以更緊密的結合。
        大概可以劃分為底下幾個項目:
	\begin{itemize}
          \item 桌面環境
          \item 應用程式的~API (也就是~Spec 1170 -- 一個統一的程式介面) --
          可能是目前為止最重要的成就。消彌了~SCO, AIX, Solaris,
          HP-UX, UnixWare 間的差異。
          \item 分散式計算環境~(OSF 的~DCE 與~SunSoft 的~ONC)
          \item 物件技術~(OMG 的 CORBA)
          \item 繪圖
          \item 多媒體
          \item 系統管理
	\end{itemize}

      \item PowerOpen Environment (POE)由~PowerOpen Association(POA) 所推動。
        是個在~PowerPC 這顆~chip 上 用的~Unix-like OS 的標準。定義了: 
	\begin{itemize}
        \item 一個~API (應用程式介面,主要源於~AIX, 符合~POSIX,
          XPG4,Motif 與~C 的標準) 與
        \item 一個~ABI (application binary i/f),這是與其他標準差異最大之處,
          POSIX, XPG4, etc.都沒有這個東西。任何符合!POE 的系統應該可以
          執行所有的~POE 軟體。
	\end{itemize}
        重要的特色:
	\begin{itemize}
        \item 建構於 PowerPC 上
        \item 與硬體匯流排無關
        \item 從膝上型電腦到超級電腦都可以用的系統
        \item 必須是個多人多工的作業系統
        \item 支援網路
        \item X windows 的擴充, Motif
        \item 是否符合標準由一個獨立的機構(POA)來測試與認定
	\end{itemize}
        AIX 4.1.1 將會符合PowerOpen。MacOS 目前不遵循 PowerOpen,也
        不打算遵循 PowerOpen。
        [以上取自於comp.sys.powerpc 的 powerpc-faq]

        IBM 在~COSE 與~POE 中都摻了一腳,這兩個組織會有怎樣的關係頗
        令人玩味。
\end{itemize}

\subsection{你所用的 Unix 是哪一種流派。}

    這個小節列出一些材料供您參考,看看能不能讓您藉此找出您所用的~UNIX
    屬於哪一種流派。由於各流派之間的程式碼或想法上都會大量交流,
    而且廠商也會自行許多修改,因此,類似「本~Unix 是~SVR2」這一類的說法,
    充其量只是統計上的敘述(但有些~SVRn 的移植除外)。

    也有許多~Unix 同時提供這兩個世界的大多數功能(不論是像~SunOS 一樣
    把這兩個世界融合在一起,或者像~Apollo 一樣把這兩個世界做嚴格的劃分)。
    所以這個小節也許沒那麼有用...。

    前一小節所列出的特性也有點兒幫助。例如,如果某一個系統有~poll(2) 但
    沒有~select(2),那它很有可能是從~SRV3 衍生出來的。同時,您也可以從~OS
    的命名或者從簽到訊息當中,找出一些蛛絲馬跡(例如~SGI 的~IRIX SVR3.3.2)
    ;此外,您也可以利用~"uname -a" 指令的輸出。找尋某些指令是否存在也是
    判斷的線索,但是直接探討核心的特性可能是比較可靠的做法。例如終端機
    初始化的方式~(inittab 或~ttys) 就是一個較可靠的指示,這比起從列印
    子系統判斷來的可靠。

\footnotesize
\begin{tabular}{|l|p{3.5cm}|p{4cm}|}
\hline
      特性          &   SVRx 的典型           & xBSD 的典型 \\
\hline 
      核心名稱      &   /unix                 & /vmunix \\
      終端機啟動    &   /etc/inittab          & /etc/ttys (only getty to 4.3) \\
      開機啟動      &   /etc/rc.d directories & /etc/rc.* files \\
      加掛檔案系統  &   /etc/mnttab           & /etc/mtab \\
      常用的shell   &   sh, ksh               & csh, \#! hack \\
      原生檔案系統  &   S5 (blk: 512-2K) \newline
                        檔案名稱~$<=$ 14 bytes & UFS (blk: 4K-8K) \newline
						 檔案名稱~$<$ 255 bytes \\
      群組          &   必須使用~newgrp(1) \newline
                        SVR4: 多重群組        & 自動加入成員 \\
      列印子系統    &   lp, lpstat, cancel    & lpr, lpq, lprm (lpd daemon) ?? \\
      終端機控制    &   termio, terminfo, \newline
                        SVR4: termios (POSIX) & termios (sgtty before 4.3reno) \newline
					        termcap \\
      工作控制      &   $>=$ SVR4             & yes \\
      ps 指令       &   ps -ef                & ps -aux \\
      多重等待      &   poll                  & select \\
      字串函數      &   memset, memcpy        & bzero, bcopy \\
      程序對映      &   /proc  (SVR4) & \\
\hline
\end{tabular}
\normalsize

      由於我們逐步跨入~90 年代末期,上述的差異已越來越不明顯了。

\subsection{請簡要介紹一些知名的 (商業/PD) Unix}

      我一點也不滿意這節的內容,很不幸地我既沒有時間也沒有文件讓它的
      內容更為完善。在此只列出一些較多人使用的 Unix, 至於其他小型或
      者非美國地區出品的 Unix 也歡迎補充,像是 Eurix。這部分還要再重
      寫一次。\footnote{這一段我也翻得不太滿意, 也要跟著重譯一次 :)}

      這個小節依字母順序列出一些廣為人知的 Unix,並且對他們的本性做
      簡要的敘述。很不幸的,在定義上它幾乎過時了...

      (只排列字母,但忽略數字與其他字元)
\begin{description}
      \item [AIX]: IBM 的~Unix, 是根據~SVR2 (最近已經出到~SVR3.2?) 以及程度不一的
        ~BSD 延伸而來, 並加上各種硬體的支援。具備特有的系統管理~(SMIT)。
         同時具有~850 與~Latin-1 CPs (Code Page, 內碼頁)。它不僅跟大多數的
        ~Unix 不相同,連~AIX 各版本之間也有許多相異之處。
         網路討論區: \url{comp.unix.aix}
	\begin{itemize}
         \item 1.x (for 386 PS/2)
         \item 2.x (for PC RTs)
         \item 3.x (for RS/6000), 分頁式核心,邏輯式容量管理程式,國際化;
         \item 3.2 新增 TLI/STREAMS.  以 SV 為核心並加上許多改進。
         \item 4.1 是最新版 (包括對 PowerPC 的支援?)
         \item AIX/ESA, 原來是在 S/370 and S/390 大型主機上執行的,它是根據 OSF/1。
           AIX 本來會成為 OSF/1 的基礎,但後來 OSF/1 選用 Mach 作為基礎。
	\end{itemize}
         我希望這個小節能寫到這裡就好 :-)

      \item [AOS] (IBM):  移植到 IBM PC RT 的 4.3BSD (這是為教育單位做的).
      它的名稱雖然 DG 專有的 OS 名稱相同,但兩者是截然不同的東西。

      \item [Arix]:  SV

      \item [A3000UX](Commodore): 以 68030 為基礎的 SVR4 Unix (?),用於 Amiga。

      \item [A/UX (Apple)]: SV 加上柏克萊的加強功能, NFS, Mac GUI.  System 6
         (也就是後來的 System 7) 可以當成~A/UX 的程式執行 (與~MachTen 相反)。
         網路討論區: \url{comp.unix.aux}
	\begin{itemize}
         \item 2.0:  SVR2 加上 4.2BSD, system 6 Mac 應用程式。
         \item 3.0 (1992): SVR2.2 with 4.3BSD and SVR3/4 延伸; X11R4,
           MacX, TCP/IP, NFS, NIS, RPC/XDR, 各類 shells, UFS 或 S5FS.
           System 7 應用程式.
         \item 4.0 將包含 OSF/1(或者就是 OSF/1)。
	\end{itemize}

      
      \item [3B1] (680x0): 根據~SV,是~Convergent 替 AT\&T 完成的。
         網路討論區: \url{comp.sys.3b1}

      \item [BNR/2]: 代表 BSD Net/2 Release? 包含 NetBSD/1, FreeBSD。

      \item [BOS for Bull's DPX/2] (680x0)
	\begin{itemize}
         \item V1 (1990): SVR3 加上 BSD 延伸 (FFS, select, sockets),
           對稱式 MP, X11R3
         \item  V2 (1991): 加上工作控制, 磁碟鏡射, C2 系統安全,
           DCE 延伸
         \item  Bull's PPC 工作站也有 BOS/X, 以及與 AIX 相容的 Unix
           至於上述兩者之間的關係則未知。
	\end{itemize}

      \item [386BSD]: Jolitz's 從~Net/2 software 移植過來的。支援~Posix, 32 位元,
           仍在~alpha 測試階段。(目前版本為~0.1 版)。
           \footnote{目前已出到 1.0 版, DDJ 要拿來賣錢的}

      \item [BSD/386] (80386): 來自~BSDI, 附原始程式 (增強的~Net2 軟體)
         網路討論區: comp.unix.bsd.
        \footnote{改名為 BSD/OS, 版本 2.x 是以 4.4BSD 為基礎}

      \item [Chorus/MiXV]: 架在~Chorus 核心之上的~Unix SVR3.2 (SVR4), 
	ABI/BCS.

      \item [Coherent] (80286): 
	另一種~Unix,與~V7 相容, 有一些~SVR2 的東東(IPC).
        ~V4.0 是~32 位元的。網路討論區: comp.os.coherent

      \item [Consensys]: SVR4.2

      \item [CTIX]: 根據 SV, 來自 Convergent

      \item [D-NIX]:  SV

      \item [DC/OSx] (Pyramid):  SVR4.

      \item [DELL UNIX] [DELL 電腦公司.]: SVR4

      \item [DomainIX]: 請參閱底下的 DomainOS。

      \item [DomainOS] (Apollo, 後來被~HP併購): 專屬的~OS; 
	上層架有~BSD4.3 與~SVR3 (使用者可以從兩層中任選
	一層、或者全選,或者都不選)。雖然現
      在已經不再發展了,但仍有些特性被~OSF/1 (與~NT) 引用。目前版本為
      ~SR10.4。SR9.* 版本的名稱為~DomainIX。網路討論區~:\url{comp.sys.apollo}

      \item [DVIX] (NT 的 DVS):  SVR2

      \item [DYNIX] (Sequent): 以 4.2BSD 為基礎

      \item [DYNIX/PTX]: 以 SVR3 為基礎

      \item [Esix] (80386):  純種的 SVR4, X11, OpenLook (NeWS), Xview

      \item [Eurix] (80?86):  SVR3.2 (德國)

      \item [FreeBSD]: 綴補過的 386bsd 0.1 程式, 並且有許多更新的工具程式。
      \footnote{以上是指 1.x 而言, FreeBSD 2.x 版是拿 4.4BSD lite 從頭改起的}

      \item [FTX]: Stratus 容錯作業系統 (68K 或 i860-i960 硬體)

      \item [Generics UNIX] (80386): SVR4.03 (德國)

      \item [GNU Hurd (?)]: 謠傳已久的軟體,來自自由軟體基金會 (FSF):
         架在 Mach 3.0 核心之上的 Unix 模擬器。有許多 GNU 的工具程式
         非常受歡迎 (emacs) 並且用於許多 PD Unix。
         \footnote{此時 GNU Hurd 已經有人使用, 目前在 alpha 測試階段,
                 可從 ftp://alpha.gnu.ai.mit.edu/... 取得}

      \item [HELIOS] (Perihelion Software): 用於 INMOS transputer 以及多種其他平台

      \item [HP-UX] (HP):  舊系統是從 S III (SVRx), 現在是~SVR2 (4.2BSD?) 加上~SV
         工具程式(他們可能很難下決心...)
\begin{itemize}
         \item 6.5: SVR2
         \item 7.0: SVR3.2, symlinks
         \item 7.5
         \item 8.0: 以~BSD 為基礎~(?) 用於~HP-9000 CISC (300/400) 與
		~RISC (800/700), 共用程式庫
         \item 9.0: 加入~DCE
         \footnote{HP-UX 已經出到 11.x 版了}
\end{itemize}

      \item [Interactive SVR3.2] (80x86): 純種~SVR3。Interactive 已經被~Sun 
	購併;它們的系統會由於 Solaris 而繼續存活嗎?
        \footnote{SunSoft 仍繼續銷售過一陣子~Interactive UNIX}

      \item [Idris]: 第一個~Unix 相容產品,是由~Whitesmith 完成的。一個小型的
	~Unix? 是給~INMOS transputer 與其他機器用的。

      \item [IRIX] (SGI):  Version 4 是由 SVR3.2, 以及許多 BSD 的東西構成的。
      Version 5.x (目前是 5.2) 是根據 SVR4。網路討論區: comp.sys.sgi。

      \item [Linux] (386/486/586): 使用~GPL (雖然不是來自~FSF) 的~Unix。
        可取得原始程式。遵循~POSIX 以及~SysV 及~BSD 的延伸。目前正進行
       ~Alpha/AXP 與~PowerPC 的移植。(目前已有版本移植到~680x0 Amigas
        and Ataris; 也有版本移植到~MIPS/4000)。
        網路討論區: \url{comp.os.linux.{admin,announce,development,help,misc}}。

      \item [MachTen], Tenon Intersystems: 當成 System 6 的一般程式來執行,沒有記憶體
         保護, 4.3BSD 環境附有 TCP, NFS。

      \item [MacMach] (Mac II): 架在~Mach 3.0 微核心之上的~4.3BSD。
	有~X11, Motif, GNU 軟體、原始程式;實驗性的~System 7 是以
	~Mach task 的方式執行。整套產品包含所有原始程式(但需要~Unix 授權)。

      \item [Mach386]: 來自 Mt Xinu。以 Mach 2.5 為基礎,附加 4.3BSD-Tahoe
         的增強功能。也有 2.6 MSD (Mach Source Distribution).

      \item [Microport] (80x86): 純種的 SVR4, X11, OpenLook GUI

      \item [Minix] (80x86, Atari, Amiga, Mac): 與~V7 相容的~Unix clone。
         產品附原始程式。目前正進行修改,使其遵循~POSIX 標準。
         適用於~PCs, 當然還有其他許許多多的機種。(如~INMOS transputer)。
         網路討論區: comp.os.minix.

      \item [MipsOS]: 有點兒~SV 的味道~(RISC/OS,現在已經不作了,
		早期有點~BSD 的味道)

      \item [more/BSD] (VAX, HP 9000/300):  Mt Xinu 的 Unix, 以 4.3BSD-Tahoe 為基礎。
         網路討論區: comp.os.xinu?

      \item [NCR UNIX]: SVR4 (4.2?)

      \item [Net/2] 磁帶 (來自 Berkeley, 1991): BSD Unix, 本質上與 4.3BSD 相容,
         裡面的程式碼完全不含 AT\&T 的程式,沒有低階的程式碼。請參閱上述的
         386BSD 與 BSD/386。

      \item [NetBSD] 0.8: 事實上只是幫 386bsd 換個新裝而已。已移植到 [34]86, MIPS,
         Amiga, Sun, Mac。 它跟 Net/2 的關係?
        \footnote{NetBSD 已經出到 1.3.3 版, 而且以~4.4BSD lite 為基礎。
                 也移植到更多的硬體平台, 如~DEC Alpha}

      \item [NEXTSTEP] (Intel Pentium and 86486, Hewlett-Packard PA-RISC, NeXT 68040):
         架在~Mach 核心之上的~4.3BSD, 具有~GUI。
	\begin{itemize}
         \item 1.x, 2.0, 2.1, 2.2, 3.0, 3.1 (舊版本)
         \item 3.2 (目前的版本
	 \begin{itemize}
            \item  Intel Pentium and 86486,
            \item  Hewlett-Packard PA-RISC,
            \item  NeXT 68040)
	 \end{itemize}
         \item 3.3 (近期內即將發表, 可能會有~Sun SPARC 的版本)
         \item 4.0 (即將發表, 可能會有~Sun SPARC 的版本)
         \item DEC Alpha 版本已經發表了
         \item OPENSTEP 遵循~OpenStep 的標準 (請參閱~Solaris)
	\end{itemize}

      \item [NEWS-OS] (Sony)
	\begin{itemize}
         \item 3.2
	\end{itemize}

     \item [OSF/1] (DEC): DEC 對~OSF/1 的移植。我認為現在 (4/94) DEC 最新
        的~Alpha AXP (64 位元機器)上應該已經有了。
        \footnote{原作者說得沒錯}

     \item [OSx] (Pyramid): 從 SysV.3 與 BSD4.3 兩者同時移植過來的。

     \item [PC-IX] (IBM 8086):  SV

      \item [Plan 9] (AT\&T): 在~1992 年發表, 是完全從頭改寫過的。
      不清楚它跟~Unix 的相近程度如何。關鍵點:分散式、非常小、多種硬體
      (~Sun, Mips, Next, SGI, generic hobbit, 680x0, PCs),
      ~C (而非謠傳的~C++)、新的編譯器、「八又二分之一」視窗系統(也非常小)、
      ~16 位元的~Unicode、架在高速網路之上的~CPU/檔案伺服器。
        \footnote{Plan9 在~95 年初又發表新版本, 從~'95 年~7 月開始賣錢,
                 CDROM: US$ 350, Document: US$ 150。
		 可參閱~\url{http://plan9.bell-labs.com/plan9/}。}

      \item [SCO Xenix (80x86)]: 給 XT (不夠快!), 286, 386 (具有 demand paging)
         使用的。現有的程式碼有很多都是從 System V 拿過來的。是一個
         穩定的產品。

      \item [SCO Unix (80x86)]:  SVR3.2 (目前已經停止取用 USL 的原始程式了)。

      \item [Sinix] [Siemens]: System V base.

      \item [Solaris] (Sparc, 80386):
	\begin{itemize}
         \item 1.0:  基本上跟~SunOS 4.1.1 是相同的,附有~OpenWindows 2.0 與
           ~DeskSet 公用程式。
         \item 1.0.1:  SunOS 4.1.2 加上多處理器的功能 (核心無多處理緒功能);
           不適用於 386。
         \item 2.0: (剛開始在 1988 年的時候是以 SunOS 5.0 的名義發表) 以 SVR4
           為基礎(具有對稱式多重處理?), 將支援 386; 附
           OpenWindows 3.0 (X11R4) 與 OpenLook, DeskSet, ONC, NIS.
           支援 a.out (BSD) and elf (SVR4) 這兩種格式、Kerberos。不附
           編譯器!
         \item Solaris 遵循最新版~(1994?) 的~OpenStep 標準
          (非~NeXT, 但具有~NEXTSTEP API)
          \footnote{Solaris 已經出到 2.7, 目前的 OpenWindows 是 3.7 版,
	         以 X11R6 為基礎。OpenStep 還沒出來,看樣子也不會出來了,
		\today}。
	\end{itemize}

      \item [SunOS] (680x0, Sparc, i386):  根據~4.3BSD, 包含許多來自~System V 
	的東西。Sun 的主要成果在於:~NFS (1984), ~SunView (1985), ~NeWS
         (1986, display postscript, 現在用於~OpenWindows), OpenLook GUI 標準,
         OpenWindows (NeWS, X11, SunView!).  網路討論區~: \url{comp.sys.sun.*}:
	\begin{itemize}
         \item 3.x:  SV IPC 套裝程式, FIFOs
         \item.0.3: 輕量級程序, 新型的虛擬記憶體, 共用程式庫
         \item.1: STREAMS \& TLI, 8-bit clean?, 非同步輸出入, ms-dos 檔案系統
	\end{itemize}
         (後續者為 Solaris -- 請參閱前幾項)。

     \item [UHC] (80x86): 純種 SVR4, X11, Motif

     \item [Ultrix](DEC):  根據~4.2BSD 再加上許多~4.3 的東西。
         網路討論區: ~\url{comp.unix.ultrix}
	\begin{itemize}
          \item 4.4 是最新版
	\end{itemize}

      \item [UNICOS] (Cray):  以 System V 為基礎。網路討論區: comp.unix.cray
	\begin{itemize}
         \item 5.x, 6,x, 7.0
	\end{itemize}

      \item [UnixWare Release 4.2] [Univel]: SVR4.2; 架在 NetWare 上。
           Univel 已經不存在了。

      \item [UTEK] (Tektronix)
	\begin{itemize}
         \item 4.0
	\end{itemize}

      \item [VOLVIX] (Archipel S.A.): 以~UNIX 為基礎的作業系統,並配上一個
      以通訊為基礎的分散式即時微核心。SVR3.2 系統呼叫、BSD4.4 檔案/網路系統
      系統呼叫~(VFS,FFS)。也有~NFS 與~X11。Vanilla VOLVIX 是給~transputer
      用的。

      Xenix (80x86):  首度出現於 Intel 硬體平台的 Unix, 以 SVR2 為基礎
         (先前是以 S III 甚至 V7 為基礎)。網路討論區: comp.unix.xenix。
\end{description}

\subsection{即時的 Unix}

      警告: 這個小節需要大幅修訂,因為有很多錯誤,而且內容也不完整。我希望
      能夠在今年冬天把它重新修訂一次。竊以為下面這些都是 UNIX,歡迎大家繼
      續補充。這個列表也包含較常見,且具有即時特性的 UNIX 系統,以及有些
      具有類似~UNIX API 的非~UNIX 系統。我不認為後者真的屬於這個範疇,但是
      既然收集了這些資料(雞肋),要捨棄掉還真有點可惜。
      請順便參考~comp.realtime。
\begin{description}
      \item [AIX]: AIX/6000 具有即時支援。

      \item [Concurrent OS] (Concurrent): 真正的 Unix,由 Concurrent 大幅更動過。

      \item [CX/UX]: 真正的~UNIX,由~Harris 大幅修改以提供即時能力與效能。遵循
             ~POSIX.4 決定版的標準。

      \item [EP/LX] (Control Data): 是移植到 R3000 的 LynxOS,以前稱為 TC/IX。

      \item [LynxOS] (Lynx Real-Time Systems, Inc): 與 Berkeley 及 SV 相容,
         從頭到尾重寫過一次(專有的), 在 SVR4 時期之前。並不是 UNIX, 但支援
         許多 UNIX I/Fs。完全的 preemptive, 固定式優先權。

      \item [MiX]: SVR4 的微核心實作,是~Chorus 提供的。

      \item [Motorola SVR4]具有即時能力。

      \item [QNX] (Quantum Software): 與~unix 相容,而且是如假包換的即時作業系統。

      \item [RTU] (Concurrent), 用於 68K 系列的機器。

      \item [Solaris 2] 具有即時能力﹖

      \item [Stellix] (Stardent); 它是~Unix, 但真的是即時作業系統嗎?

      \item [Venix/386]: 交談式的~SVR3.2 附加即時延伸。

      \item [VMEexec] (Motorola): 不是~Unix, 但也用了一些跟~Unix 相同介面。

      \item [VxWorks] (Wind River Systems):~跟~Unix 相同的地方非常非常地少,
         但還是有一些介面跟~Unix 相同(但檔案系統不同)。
         網路討論區~\url{comp.os.vxworks}。
\end{description}

        (對於以下的系統,我一無所知)
\begin{itemize}
      \item Convex RTS

      \item REAL/IX (AEG)

      \item Sorix (Siemens)

      \item System V/86 (Motorola)

      \item TC/IX (CCD)

      \item Velocity (Ready Systems):
\end{itemize}


\subsection{Unix 用語彙編}

    底下提供~Unix 系統相關的概念與構成要素之簡單定義。
\begin{description}
    \item[Chorus] 訊息傳遞的微核心, 有可能成為未來~SV 的基礎。SVR4 已經可以
         架在~Chorus 上面跑了(二進位相容)。\footnote{結果~Chorus
	已經快不見了:-) 有興趣請看~\url{http://www.chorus.com}}

    \item [CORBA] (Common Object Request Broker Architecture)
	\footnote{http://www.omg.org/}

    \item [COSE](Common Open Software Environment) [Sun, HP, IBM]: common look and
         feel (Motif -- Sun will let OpenLook fade away), common API.
         Reaction against Windows NT. See section 6.4 above.

    \item[DCE](分散式計算環境 -- Distributed Computing Environment, 來自 OSF):
         包含 RPC (Apollo's NCS), 目錄服務 (區域性使用是以 DNS 為基礎,
         全球性使用則是以 X.500 為基礎), 時間, 系統安全, 以及處理緒 (thread)
         服務, DFS (分散式檔案系統 -- distrib. file system), ....
         與 OS 種類無關。

    \item [DME] (分散式管理環境 -- Distributed Management Environment, 來自 OSF):
         未來。

    \item [DO] (Distributed Objects [Enterprise]): ???.

    \item [FFS] (高速檔案系統 -- Fast File System): 於 1983 年出自 Berkeley。
         與 SunOS 的 UFS 等義 (或完全相同?)。具有一些像是
         磁柱群組(cylinder groups), 碎屑 (fragments) 等觀念。

    \item [Mach] 來自 CMU (卡內基美濃大學)的新式核心,許多 Unix 以及其他 OS
        都以此為基底(例如 OSF/1, MacMach):
         - 2.5: 集成式(monolithic)核心, 附有 4.2BSD
         - 3.0: 微核心,附有在使用者空間的 BSD Unix 伺服器 (以及其他 OSs,
           例如 MS-DOS)
         網路討論區: comp.os.mach

    \item [MFS](記憶體檔案系統 -- Memory File System):

    \item [NeWS](Network extensible Window System), 來自 Sun?: 以 PostScript 為基礎,
         網路化, 工具程式組(甚至客戶程式)是在伺服器端載入。是 OpenWindows 的
         一部份。

    \item [NFS] (網路檔案系統 -- Network File System):  由 Sun 貢獻給 BSD 的禮物,
         無狀態式伺服器。

    \item [ONC] (開放式網路計算 -- Open Network Computing): 來自 Sun(?), 包含 RPC, 
        名稱服務(NIS 也稱為 YP), NFS, ...(可以在許多 Unix 或其他 OS 當中發現)

    \item [OpenStep] [NeXT, Sun]: ???

    \item [PowerOpen]: 既是一種標準, 也是一個推廣 PowerPC 的組織。
         成員包括 IBM, Apple 與 Motorola; 還有其他廠商?
         請參閱之前的 6.4 節。

    \item [PowerPC (PPC)]: 一種 RISC CPU 晶片 [IBM, Motorola].

    \item [RFS] (遠端檔案系統 -- Remote File System):  SV, 會記憶狀態的伺服器,
         與~NFS 不相容。

    \item [RPC] (遠端程序呼叫 -- Remote Procedure Call): 高階的~IPC (Inter-Process
         communication) 機制。有兩種流派。
         - ONC: 架在~TCP 或~UDP 之上(後來也架在~OSI 之上), 使用~XDR 來
           做資料的編碼。
         - DCE: 具有不同的~RPC 機制 (根據 Apollo 的 NCS)

    \item [S5 FS]:  原生於~System V 的檔案系統, 一個區塊的大小為~512 到~2K。

    \item [sockets]:  BSD 使用網路的介面機制 (請與~TLI 比較)。

    \item [STREAMS]:  一種用於訊息傳遞的核心機制, 是從 SVR3 開始有的, 它提供了
         一個非常適用於發展協定的介面。

    \item [TFS] (半透明檔案系統 -- Translucent File System): Sun, COW 應用到檔案上。

    \item [TLI] (Transport Library Interface):  SV 傳送服務(TCP, OSI) 的介面,
         UI 也定義一個 APLI (ACSE/Presentation Library Interface --
         呈現層程式庫介面)

    \item [UFS] (?): BSD 的原生檔案系統,就是我們在 SunOS 看到的那一種,區塊
         大小為 4K to 8K, 具有磁柱群組與碎屑的觀念。

    \item [XTI] (X/Open Transport Interface):  加強型的 TLI

    \item [X11]: MIT 所發展的以畫素組成的視窗系統
\end{description}
\end{CJK*}
\end{document}
