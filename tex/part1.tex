\documentclass{letter}
\usepackage{CJK}
\usepackage{url}

\begin{document}
\begin{CJK}{Bg5}{ming} 
這七篇文章包含一些在 comp.unix.questions 和 comp.unix.shell 常見到的問
題。請不要再問這些問題,因為這些問題已經回答過太多次了。但也請不要因為
有人問這些問題而發火,因為他們可能尚未讀過這些文章。

This collection of documents is Copyright (c) 1994, Ted Timar, except 
Part 6, which is Copyright (c) 1994, Pierre Lewis and Ted Timar.

All rights reserved. Permission to distribute the collection is 
hereby granted providing that distribution is electronic, no money is 
involved, reasonable attempts are made to use the latest version and 
all credits and this copyright notice are maintained.

Other requests for distribution will be considered.

All reasonable requests will be granted.

中文翻譯 by \url{{chenjl,freedom,jjyang}@csie.nctu.edu.tw}
若您對中文翻譯有任何意見請發 e-mail 給 \url{cfaq@csie.nctu.edu.tw}

我們希望這些文件中的資訊能對你有所幫助﹐但是並不保証是正確的。若發生損
害請自行負責。

您可以在 rtfm.mit.edu 的 pub/usenet/news.answers 找到包括此文件在內的
許多 FAQ。 在此目錄下的 FAQ 的名字可在文章的頂端的 "Archive-Name:" 
那一行找到。

/* 譯注: 在台灣請用 \url{ftp://NCTUCCA.edu.tw/USENET/FAQ},若您人在交大的話 
         \url{ftp://ftp.csie.nctu.edu.tw/pub/Documents/FAQ}是從 CCCA mirror 
來的 */

此一 FAQ 是以"unix-faq/faq/part[1-7]" 為名。

這些文章大約分成:
\begin{quote}
      1.*)一般性的問題

      2.*)初學者可能會問的基本問題

      3.*) 中級的問題

      4.*) 自以為已經知道所有答案的人可能會問的高級問題

      5.*) 關於各種 shell 的問題

      6.*) 各式各樣的 Unix

      7.*) An comparison of configuration management systems (RCS, SCCS).

\end{quote}

這篇文章回答了以下問題:
\begin{description}
      \item [1.1)] 誰幫助你將做出這個 faq 的?
      \item [1.2)]  當有人提及'rn(1)' 或 'ctime(3)' 的時候,刮弧裡的數目意謂
	    什麼呢?
      \item [1.3)]  一些奇怪的 unix 指令名字的由來?
      \item [1.4)]  "comp.unix.questions" 和 "info-unix" mailing list 之間的
	    gateway 是如何運作的?
      \item [1.5)]  請建議一些有用的 Unix, C 相關書籍.
      \item [1.6)]  以前在這 FAQ 中的發音表到那去了?
\end{description}
若要找問題 1.5 的答案, 用 regular expression 往前找 "\^1.5)" 即可。

因為這些都是正當合理的問題, 所以在 \url{comp.unix.questions} 或是 
\url{comp.unix.shell} 中。每隔一陣子, 就會有這些問題與答案出現, 緊接著就會
有人對同樣問題一再出現發牢騷。關於 UNIX 代表啥呢? 請參考每月 post 在 
\url{news.announce.newusers} 中名為 "Answers to Frequently Asked Questions" 
的文章。

因為 Unix 有太多不同的種類了, 所以很難保證此文件所提供的答案必然會有
用。在嘗試本文件提供的作法前, 請先讀讀你所使用系統的手冊。若你對答案
有任何建議或更正, 請送 email 給 tmtaimar@isgtec.com.

---------------------------------------------------------------------

Subject: Who helped you put this list together? \\
Date: Thu Mar 18 17:16:55 EST 1993

1.1)  誰幫助你將做出這個 faq 的?

	這個文件是最早的幾個 FAQ 之一, 編纂於 1989 年七月。整個編輯工作幾
	乎都是 Steve Hayman 做的。 我只是接管這份文件的維護。

	我們非常感謝 Usenet 讀者提出的問題,反應,更正和提議。

	特別感謝 Maarten Litmaath, Guy Harris 和 Jonathan  Kamens,他們都
	貢獻良多。

	這個文件中的第 5 部份(shells)幾乎完全是 Matthew Wicks 
	\url{<wicks@dcdmjw.fnal.gov>} 寫的。

	這個文件中的第 6 部份(各種 Unix)幾乎完全是 Pierre (P.) Lewis 
	\url{<lew@bnr.ca>}寫的

	我儘可能的把每個問題的作者與最後更新的日期放在問題的開端。很不幸
	地,因為我最近才開始這樣做, 許多的資訊已遺失。我也疏於保存為問題
	補充最新資料者的名單。對那些有貢獻而沒有得到應有的榮譽者, 只能在
        此致歉。

	我將此文件轉為 *roff 格式(ms 與 mm 兩種 macro 都有)。Andrew 
	Cromarty把它轉換成 Texinfo 格式。這些格式化的版本可由 anonymous 
	ftp 於 \url{ftp.wg.omron.co.jp} 的 pub/unix-faq/docs 取得.

------------------------------

Subject: When someone refers to 'rn(1)' ... the number in parentheses mean?\\
Date: Tue, 13 Dec 1994 16:37:26 -0500

1.2)  當有人提及'rn(1)' 或 'ctime(3)' 的時候,刮弧裡的數目意謂什麼呢?

	看起來像是 function call, 不過, 不是。 這些數目字是指 Unix 手冊中
	文件所在的章節。當你打 "man 3 ctime" 時, 表示是要查閱在第 3 節中 
	ctime 的內容。

	傳統 Unix 手冊的分節法:

	        1       User-level  commands
	        2       System calls
	        3       Library functions
	        4       Devices and device drivers
	        5       File formats
	        6       Games
	        7       Various miscellaneous stuff -  macro packages etc.
	        8       System maintenance and operation commands

	有一些 Unix 版本會使用非數字的章節名字。舉例來說﹐Xenix 的指令用
	"C" 而函數用"S"。 一些比較新的 Unix 版本得用 "man -s\# title" 而非 
	"man \# title"。 
	
	/* 
             譯注: 如 Solaris 2.x   
	*/
      
	每一節都會有一個簡介, 若以 \# 代表節數, "man \# intro" 就可以讀第 \# 
	節的簡介。

	有時為了區別指令和相同名字的常式或系統呼叫數字是必需的。 舉例來說
	,你的系統可能有"time(1)", 有關 'time' 這個測量所費時間指令的說明
	,也有 "time(3)", 關於 'time' 這個用來決定目前時間的副常式的說明。 

	你可以用  "man 1 time" 或 "man 3 time" 來選擇要看哪一 "time" 的說
	明。

	或許你系統會有其它的章節或在細分的次章節(subsection),像 Ultrix 就
	有 3m, 3n, 3x 與 3yp。
 
------------------------------

Subject: What does {some strange unix command name} stand for? \\
Date: Thu Mar 18 17:16:55 EST 1993


1.3)  一些奇怪的 unix 指令名字的由來

    awk = "Aho Weinberger and Kernighan"

	這個語言以作者 Al Aho, Peter Weinberger 和 Brian Kernighan 的
	姓來命名。

    grep = "Global Regular Expression Print"

	grep 源自 ed。用來列印所有符合某 pattern 的指令

	g/re/p

	"re" 代表 regular expression

    fgrep = "Fixed GREP".

	fgrep 祇找固定的字串。"f" 不是代表 "fast" - 事實上, "fgrep 
	foobar *.c" 通常比 "egrep foobar *.c" 來得慢(有點意外吧, 不信
	的話, 自己試試嘍)。

	儘管如此,Fgrep 仍然有可取之處,在檔案中搜尋字串的時候,Fgrep 
	能處理的字串數目較 egrep 多。

    egrep = "Extended GREP"

	egrep 用比 grep 更 fancy 的 regular rexpression。許多人只用 egrep,
        因為它用的 algorithm 比 grep 或 fgrep 用的要好一些, 而且通常是三個
        程式中最快的。

    cat = "CATenate"

	catenate 是一個滿曖昧的字,在這裡的意思是"把它連成一串", 這就是
	"cat" 這個指令對一個或多個檔案所做的處理。

	請不要跟 C/A/T 混淆了,C/A/T 是指電腦輔助排版系統(Computer 
	Aided Typesetter)。

    gecos = "General Electric Comprehensive Operating Supervisor"

	不過,通用電器 (GE) 的大型系統部門賣給 Honeywell 的後, 
        Honeywell 就把 GECOS 的 E 拿掉了。

	目前 Unix 的密碼檔裡面仍保有 "pw\_gecos" 這個欄位。這個名字是
	從古早的年代沿用過來的。

	Dennis Ritchie 曾經說過:

	    "有時候我們會把印表輸出或整批工作丟到 GCOS 機器。密碼檔裡
            面的 gcos 欄位是用來隱藏 \$IDENT 卡片上的資訊,這樣做不夠優雅"。

	    /* 
             譯注: 這句話不知是甚麼意思, 只好照字面翻譯之
	    */

    nroff = "New ROFF"
    troff = "Typesetter new ROFF"

	這些字都是從"roff"衍生的, roff 是重寫 Multics 上的 runoff 程式得來
	的 (runoff 的意思就是"印出文件")。

    tee = T

	這是管線工人的術語,代表 T 型的管線分叉器。

    bss = "Block Started by Symbol" (由符號啟始的區塊)

	Dennis  Ritchie 曾說過:

	    這個縮寫也許有其他說法,但事實上我們採用這個縮寫的本意是 
            "Block Started by Symbol"。它是 FAP 上的虛擬指令,FAP 
            (Fortran Assembly [-er?] Program) 是指 IBM 704-709-7090-7094 
	    這種機型的組譯器。這個指令可定義自己的標號,並且預留一定數目
	    的字組空間。還有另一個虛擬指令 BES,是 "Block Ended by 
   	    Symbol",跟 BSS 指令幾乎一樣,不同點在於標號是定義在預留字組
	    空間尾端的位址 + 1 的地方。在這些機器上,Fortran 的陣列是以反
	    方向儲存,而且陣列的索引是從 1 算起。

	    這種用法是合理的,因為這跟 UNIX 上標準的程式載入器一樣,程式
	    碼當中並非真的放入這一整塊預留空間,而是先用一個數目表示,在
	    載入時才真的把所需的預留空間定出來。

    biff = "BIFF"

	這個指令是用來設定當您有新郵件進來時,是否要通知您。這是柏克
	萊大學校園內一隻狗的名字。

	    我可以確定這個名稱的起源,如果您有興趣的話,Biff 是 Heidi 
	    Stettner 養的寵物,想當年 Heidi (還有我,跟 Bill Joy) 都還是 
	    UCB 的研究生時,早期的 BSD 版本還在發展中。Biff 受到流連於 
	    Evans Halls 這些人的喜愛,也因為它會對前來的郵差吠叫而聞名;因
	    此就以 biff 當作指令的名稱。
	    (這是卡內基美濃大學的 Eric Cooper 證實的)

    rc (像是 ".cshrc" 或 "/etc/rc" 中的 rc 這兩個字母) = "RunCom"

	"rc" 源自麻省理工學院在 1965 年發展的 CTSS系統中的 "runcom"。
	Brian Kernighan 與 Dennis Ritchie 告訴 Vicki Brown 說: 

	    相關文獻寫著這麼一段話: 'CTSS 具有從檔案中取出一系列命令
	    來執行的功能;這功能稱為 "run commands" 又稱為 "runcom",
	    而這種檔案稱為一個 runcom (a runcom)。'

	    Unix 中的 "rc" 就是這種用法的殘跡。

	此外 "rc" 也是Plan 9 作業系統 shell 的名字。

    Perl = "Practical Extraction and Report Language"
    Perl = "Pathologically Eclectic Rubbish Lister"

	Perl是 Larry Wall 所發展的一種相當受歡迎的語言, Perl 在處文字, 
	process,與檔案時非常便利,可以說是兼得 shell 與 C 之長。想知道
        更多關於Perl 的訊息,請看 Usenet  newsgroup  comp.lang.perl。

    Don Libes 的 "Life with Unix" 一書裡有更多這類的珍聞軼事。

------------------------------

Subject: How does the gateway between "comp.unix.questions" ... work? \\
Date: Thu Mar 18 17:16:55 EST 1993

1.4)  "comp.unix.questions" 和 "info-unix" mailing list 之間的 
      gateway 是如何運作的?

      "info-unix" 與 "unix-wizards”分別是 \url{comp.unix.questions} 和 
      \url{comp.unix.wizards} 的 mailing-list 版。Mailing list 與 
      newsgroup 的內容應該是相同的

      要加入或退出任一個 mailing list, 請送 email 給info-unix-quest@brl.mil 
      或 unix-wizards-request@brl.mil。切記要加入或退出 mailing list 時是送
      給 "*-request@brl.mil" 。也請您耐心等候因為不會馬上有回應。
        
      底下就是 mailing list 的維護者 Bob Reschly 所提供關於這些 mailing list 
      的詳細細節。

      ==== postings to info-UNIX and UNIX-wizards lists ====

      我個人不對送到 mailing list 的內容作管制,任何送到這個 mailing 
      list 的信件都會貼到相對應的 news group。BRL 只是單純的轉送。在
      此 mailing list 上的 Internet 使用者,要送文章到 newgroup 上時
      請送到 info-UNIX 或 UNIX-wizards, '-request' 是要的信是要送給 
      mailing list 的維護者看的。

      在此 mailing list 上的 Internet 使用者會收到兩類的訊息,一種是
      單獨一篇的討論文章,另一種則是集結多篇討論精華的文摘。從 
      Internet 或者 BITNET(透過 Bitnet ←→ Internet 轉換程式)寄往 
      BRL 的訊息都會轉發一份給 mail list 當中的每一個使用者。


      從 USENET 發出來的文章則以每天彙整一次的方式寄給 mailing list 
      當中的所有成員。BITNET 的網路交通跟 Internet 的網路交通很像。
      主要的不同點在於:對於 mailing list 內所有 BITNET 的收件人,我
      只要維護一個電子郵件地址,讓這個地址所在的收件程式維護所有收件
      人的資訊,並且自動轉寄所有文章給 mailing list 上的每一個訂閱者
      即可。

      在 USENET 上的訂閱者只會讀到各自獨立的訊息,所有發自 Internet 
      的訊息則轉送至我們位於 USENET 上的機器,然後貼至合適的討論區。
      很不幸地,這些透過轉換程式貼出去的文章,發件人會變成 news@brl-adm,
      這是轉換軟體目前尚未解決的先天限制。

      至於讀者群方面,USENET 是一個擁有廣大讀者群的地方,我估計約有數
      千部主機與數萬名使用者參與 USETNET。BRL 所維護的主要 list 約有 
      250 個,大約有百分之十是本地的轉送 list。我不太清楚 BITNET 方面的
      轉送數目,不過如果要讓我猜的話,數目大約跟主要 list 一樣。平均一個 
      list 在一個星期內要送出 150K 到 400K 的資料。


------------------------------

Subject: What are some useful Unix or C books? \\
Date: Thu Mar 18 17:16:55 EST 1993

1.5)  請建議一些有用的 Unix, C 相關書籍。

	Mitch  Wright (mitch@cirrus.com) 維護一份 Unix 和 C 相關書籍一覽
	表,裡面包含簡介與短評。目前在他的表上有 167 本書ftp.rahul.net 
	(192.160.13.1) 的 "pub/mitch/YABL/yabl" 就是這份一覽表。 要加入新
	的內容或提供建議送 email 給 mitch@cirrus.com。

	Samuel Ko (kko@sfu.ca) 維護一份 Unix 相關書籍表。這個列表只包含推
	薦書,因此比較短。 這份表是分類的列表, 如果你正在尋找特定條件類
	型的書, 這份表無疑是較為合適的。rtfm.mit.edu 的
	``\url{pub/usenet/news.answers/books/unix}''  就是此表。
	要加入新的內容或提供建議送 email 給 kko@sfu.ca。

	如果你不能使用 anonymouse ftp,  email 到 "ftpmail@decwrl.dec.com" 
	信的內容就寫 "help",然後你就會收到一份教你如何以 email 取得 
	anonymous 的信件。

------------------------------

Subject: What happened to the pronunciation list ... ? \\
Date: Thu Mar 18 17:16:55 EST 1993

1.6)  以前在這 FAQ 中的發音表到那去了?

	當 1989 本文件開使時, 它包含了一份 Carl Paukstis 
	<carlp@frigg.isc-br.com> 原作,由 Maarten Litmaath 所維護的一份包
	羅萬象的發音表。後來它功成身退了,因為發音與 "Unix questions" 這
	個主題不是真的有關。

	若你碰到一些不知該怎麼讀的字, 請參考 Eric S. Raymond
	\url{<eric@snark.thyrsus.com>}所維護的 Jargon。

        /* 譯注: ftp://ftp.csie.nctu.edu.tw/pub/unix/GNU/jarg400.txt.gz 是 
                 普通文字版 Jargon */

	若你還是堅持要以前那份發音表, ftp.wg.omron.co.jp (133.210.4.4) 的 
	``\url{pub/unix-faq/docs/Pronunciation-Guide}'' 就是啦。

------------------------
unix / faq 第 1 部份結束

\end{CJK}
\end{document}
